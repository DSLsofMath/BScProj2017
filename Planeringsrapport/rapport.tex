% CREATED BY DAVID FRISK, 2018

% IMPORT SETTINGS
\documentclass[12pt,a4paper]{article}
\usepackage[utf8]{inputenc}

\setlength{\parindent}{0pt}
\setlength{\parskip}{\baselineskip}
\usepackage[a4paper, margin=1in]{geometry}

\usepackage[noconfigs, swedish]{babel}
\usepackage{hyperref}

\usepackage{titling}
\newcommand{\subtitle}[1]{%
  \posttitle{%
    \par\end{center}
    \begin{center}\large#1\end{center}
    \vskip0.5em}%
}

\begin{document}

\pagenumbering{gobble}

\title{Learn You a Physics for Great Good}
\subtitle{Using domain specific languages to teach physics}
\date{\today}
\author{Björn Werner\\ Erik Sjöström \\ Johan Johansson \\ Oskar Lundström}

\maketitle

% TABLE OF CONTENTS
\newpage
\tableofcontents

% START OF MAIN DOCUMENT
\newpage
\setcounter{page}{1}
\pagenumbering{arabic}			% Arabic numbering starting from 1 (one)
%\setlength{\parskip}{0pt plus 1pt}

\section{Bakgrund}

\subsection{Varför}

%PaJa: formulera om "allmänt känt att betygsstatistiken": kanske något stil med "Många D-studenter har problem med ..."?
Det är allmänt känt att betygsstatistiken för datastudenter som läser fysik är ovanligt låg.
%PaJa: "postulerar" känns inte bra här. "tror"? Jag är också osäker på om "metoden" verkligen skiljer sig drastiskt. Tydliggör vad ni menar, och vilka andra kurser ni jämför med.
Vi postulerar att detta beror på metoden som används för att lära ut fysik skiljer sig drastiskt från hur andra kurser lärs ut inom data.
%PaJa: OK
Och om det fanns ett sätt att kombinera det material som ska läras ut med de pedagogiska verktyg som används inom datakurser så kan vi underlätta lärandet.
%PaJa: kan nog behöva förklaras mer.
Vår hypotes är att om man kan ``lära'' en dator fysik så har man själv förstått materialet.

%PaJa: var är bilagan? Olämpligt "namn".
Se bilaga: Möte med Åke bråke

\subsection{Vad?}

Vi vill med detta projekt skapa ett pedagogiskt läromaterial som med hjälp av en eller flera DSL:er ska fungera som ett komplement till de kurser i fysik som riktar sig till studenter inom datateknik.

\subsection{Relevant för?}

Detta läromaterial kommer främst att rikta sig till studenter inom data som tar en fysikkurs. Men det kan också ses som relevant läsning för en student inom fysik som är ute efter en inkörsport till funktionell programmering.
%
Förhoppningsvis blir det också relevant för de som är intresserade av DSL:er i stort, pedagoger och föreläsare inom de berörda områdena
%
%PaJa: angående "skifta läroplanen" - det är oklart vad ni menar här. Formulera om med tanke på rubriken "relevant för" - programledning kanske vill utveckla / förbättra programmet, etc.
och kanske till och med programledningen som ser vår rapport som ett skäl att skifta läroplanen.

\subsection{Relaterat till?}

%PaJa: bra med referenser - men gör en .bib-fil och citera ordentligt.
Detta arbete kan relateras det arbete som redan utförts inom detta område t.ex. \textit{Structure and Implementation of Classical Mechanics}\cite{SICM}.
%PaJa: citera mer specifikt (kursen, artikeln, eller båda). [Det är heller inte specifikt bara "analys".]
Och även det arbete som Patrik Jansson utfört med att brygga gapet inom analys.
%PaJa: OK, men bör kanske formuleras om för att tydliggöra.
Vi skulle också gärna se att vårt arbete kan relateras till den större diskussionen om varför matte och fysik lärs ut inom data.


\section{Syfte}

Syftet med projektet är att skapa domänspecifika språk för fysik samt ett tillhörande läromaterial. Läromaterialet ska både förklara ''ren'' fysik och parallellerna mellan de domänspecifika språken och fysiken. Förhoppningen är att väcka intresse för fysik hos datateknologer genom att presentera fysik från ett annat perspektiv.

%Syftet är en kort beskrivning av uppdraget och vilket resultat som uppdraget
%ska leda till.
%Projektets syfte är att implementera ett antal DSLs i ett par olika discipliner, samt
%en dokumentation över användningen av dom.
%
%\begin{itemize}
%    \item 1-2 meningar
%    \item Vad ska resultatet vara?
%\end{itemize}

\section{Problem/Uppgift}

%Oskar: Detta är min bild. Rätta om jag fått något om bakfoten
Slutprodudukten är tänkt att bli en handledning till fysik mixat med domänspecika språk för fysik. Handledningen ska vara en brödtext varvat med programkod och ska att ha en lättsam stil, i likhet med \textit{Learn You a Haskell}\cite{LYAH} som är en handledning för Haskell. Vår tanke är att presentera fysik parallellt med att vi bygger upp domänspecika språk som används till fysik. Vi kommer även i texten presentera exempel på fysikaliska problem och sedan visa hur man kan modellera och lösa dem i de domänspecifika språk vi skapat.

De uppgifter vi har framför oss är därmed:

\begin{itemize}
	\item Hitta de områden inom \textit{Fysik för ingenjörer} som datateknologer har svårt för.
	\item Inläsning av...
	\begin{itemize}
		\item ...fysik.
		\item ...domänspecifika språk.
		\item ...liknande projekt för att få inspiration.
	\end{itemize}
	\item Skriva handledning och parallellt implementera egna domänspecifika språk.
\end{itemize}


Några frågor uppträder och de vill vi kunna besvara när det här projektet är klart:
%TODO: Vissa frågor kanske ska tas bort eller omforumerlas nu när vi bytt fokus
\begin{itemize}
    \item Kan man skapa domänspecifika språk för fysik?
    \item Hur brett/djupt kan vi komma med våra domänspecfika språk?
    \item Kan man använda våra domänspecika språk för att lära ut?
    \item Kan vi skapa ett läromaterial utifrån våra domänspecifika språk?
    \item Kommer vår handledning främja lärandet?
    \item Är djup eller ytlig inbäddning bäst ur ett pedagogiskt syfte?
\end{itemize}

\section{Avgränsningar}

Projektet kommer enbart behandla den fysik som ingår i kursen \textit{Fysik för ingenjörer}. Denna avgränsningen valdes dels för att det är så mycket fysik gruppmedlemmarna kan, dels för att det är den kursen detta projekt kan blir mest relevant för.

\textit{Fysik för ingenjörer} behandlar grunderna inom de tre områdena mekanik (inklusive stelkroppsmekanik), termodynamik och vågrörelselära. Vi har valt att i första hand prioritera mekanik, för att senare i mån av tid även behandla termodynamik och vågrörelselära. Dessutom kommer de områden datateknologer haft svårt för prioriteras. Även om projekets direkta syfte inte är att förbättra statistiken, är vår förhoppning att genom att behandla de svåra delarna mer ingående, kommer vår alternativa presentation ändå indirekt kunna underlätta förståelsen inom dessa områden.

%TODO: Koppla till etik!


% Under avgränsningar talar man om vad man inte behandlar.
% Projekter avgränsar sig till exempelvis att implementera DSLs för samtliga områden
% inom kursen Fysik för ingengörer.

% \begin{itemize}
%     \item Vilka delar av det övergripande syftet som ej ska med
% \end{itemize}

\section{Metod/Genomförande}

%TODO: Gäller detta fortfarande?
Som ett genomgående tema vill vi arbeta in återkoppling med FysikÅke och DSLPatrik så att vi vet att vi håller oss på banan och inte gör det för svårt för oss själva eller potentiella studenter.

Den övergripande planen är att börja med att läsa in oss på fysik, domänspecika språk och liknande projekt. Detta för att få en grundläggande uppfattning av hur projeket kan tänkas se ut. Därefter kommer handledningen att skrivas. Eftersom domänsepcika spårk kommer presenteras varvat i handledningen kommer därför också de domänspecika språken skapas parallellt med skrivandet. Vi vet ännu inte hur vi bra kan fördela arbetet mellan gruppmedlemmarna. En möjlighet, beroende på tidsbehovet, är att en/två skriver om mekanik och en/två om termodynamik eller våglära. Under skrivandets gång kommer en del inläsning behöva göras parallellt. Det kan handla både om fysik och domänspecika språk, till exempel att jämföra vår implementation med likaratade implementationer, för att se att vi håller oss på banan.

För att hitta de områden datateknologer har problem med i \textit{Fysik för ingenjörer} kommer vi prata med kursens föreläsare, med DNS (Datateknologsektionen på Chalmers) samt reflektera över de delar vi själva tyckte var svåra.

De primära källorna till inläsning av fysik kommer att vara kursboken samt föreläsarens egna material i form av anteckningar och övningsuppgifter. Till vår hjälp för att förstå hur man bra kan skapa domänspecika språk inom fysik har vi boken \textit{Structure and Interpretaton of Classical Mechanics}\cite{SICM}.

Se bilaga $\lambda$ för en mer detaljerad lista över hur vi tänkt lägga upp arbetet.

% Metodkapitlet ska beskriva hur man avser att lägga upp arbetet. Detta
% omfattar bland annat arbetsgång, design av experiment och användning av olika
% datainsamlingsmetoder. Ett metodkapitel ska, i idealfallet, vara så utförligt att
% vem som helst som har vissa baskunskaper inom området ska kunna utföra arbetet
% på det sättet som har beskrivits i rapporten och nå samma resultat. Att beskriva
% metoden är viktigt för att uppdragsgivaren ska kunna bedöma om man kan nå målet
% på det föreskrivna sättet. Det är därför också viktigt att man förklarar varför den
% valda metoden ger ett tillförlitligt resultat.

\subsection{Avslutning}

För att avsluta vårt projekt så vill vi återkoppla med studenter inom data som både har läst fysikkursen och de som inte har läst den för att på så sätt få deras perspektiv på det hela. Detta skulle även ge en fingervisning om hurvida vårt resultat är tillförlitligt. Lämpliga frågor är då:

\begin{itemize}
    \item Tror du att detta \textbf{hade} hjälpt dig när du studerade fysik?
    \item Tror du att detta \textbf{kommer} att hjälpa dig när du studerar fysik?
    \item Har vi missat nåt?
    \item Vad hade vi kunnat göra bättre?
\end{itemize}

I mån av tid så vill vi använda denna återkoppling för att förfina vårt läromaterial. Med den större förhoppningen att vi kan producera ett läromaterial som är bra nog för att publiceras.

\section{Samhälleliga och etiska aspekter}

En positiv etisk aspekt på projektet är att vi valt att prioritera de områden som varit svåra för datateknologer. Det ger med andra ord ett potentiellt nyttovärde för datateknologer som läser \textit{Fysik för ingenjörer}.

En negativ etisk aspekt är att nyttovärdet enbart gynnar de som är bra på Haskell och funktionell programmering. För att kunna tillgodogöra sig den eventuella kunskapsvinning från detta projeket krävs det att man är bra på Haskell eftersom allting uttrycks i det programmeringsspråket.

I likhet med de flesta kunskapsinsamlande projeket, även om något så ''oskyldigt'' som matematik och grundläggande mekanik, uppkommer frågan om i vilka syften kunskapen kan användas till. Den kunskap vi hittar är i sig inte farlig eller missgynnande för någon. Däremot kan den användas av andra till illvilliga syften. Det skulle vara möjligt att spinna vidare på detta projeket och skapa till exempel bomber av olika slag. Detta är en aspekt som är viktig att ha i åtanke.




----------gamla nedan

Nedan följer svar på de samhälleliga och etiska frågor som ställts i anvisningarna till projektplanen-

1) Vilka etiska  aspekter (värden) är relevanta för projektet?
\[
\emptyset
\]

Vi tycker inte det finns några negativa eller positiva etiska konsekvenser av detta projekt. Vi ska konstruera ett läromaterial om grundläggande fysik och har svårt att se hur det kan påverka någon individ eller grupp negativt.

2) Hur kan vi genomföra vårt projekt för att undvika etiska problem med vår metod?

Vi kommer att kommunicera med vår handledare, en föreläsare och studenter som testar vårt material. Givet att vi är allmänt trevliga ser vi inga etiska problem med denna metod.

3) Vad kan det finnas för nytta eller etiska problem med det sannolika resultatet (utfallet) av projektet som man bör ta hänsyn till?

Främjandet av fysikinlärning för datastudenter kan möjligtvis leda till positiva aspekter för samhället i stort, om inte annat för att utbildning blir lättare att ta till sig. Förhoppningsvis kan vi också göra smärre nytta för forskningen inom detta område (DSL för pedagogik?).

De enda dilemman där de etiska aspekterna skulle kunna äventyras skulle antagligen vara om vårt läromedel utnyttjades för att lära ut kärnfysik till studenter i länder med kärnvapenprogram \url{http://sverigesradio.se/sida/artikel.aspx?programid=3182&artikel=2980763}.
   Detta lär inte vara något vi behöver beakta i vårt läromedel, då kursen fysik för ingenjörer inte innefattar någon kärnfysik.

4)  Vilka berörs av projektets genomförande eller av det sannolika resultatet (utfallet) av projektet? Hur berörs de? Finns det etiska problem kopplat till detta som man bör ta hänsyn till?

Se ovanstående svar.

\section{Tidsplan}

Tidplanen presenteras lämpligen i form av ett Gantt-schema.
(Kommer läggas till då vi har uppskattat tidsomfattningen).

\begin{itemize}
    \item Vad är när saker ska göras
    \item Var konkret och detaljerad
\end{itemize}

\textbf{Milstolpar}

\begin{itemize}
    \item 2018-02-06 - Innehåll och medlemmar bestämt till halvtidsredovisning
    \item 2018-02-09 - \textbf{Planeringsrapport färdig och inlämnad}
    \item 2018-02-16 - Manus och presentationsmaterial till halvtidsredovisning skapat
    \item 2018-02-23 - Genrepat halvtidsredovisning
    \item 2018-02-27 - \textbf{Halvtidsredovisning}
    \item 2018-02-03 - \textbf{Egen utvärdering på blankett till handledaren}
    \item 2018-03-23 - \textbf{Bestämma rapportspråk}
    \item 2018-05-08 - \textbf{Inlämning av poster för tryckning}
    \item 2018-05-15 - \textbf{Utställning}
    \item 2018-05-16 - \textbf{Engelsk titel angiven}
    \item 2018-05-21 - \textbf{Inlämning opposition}
    \item 2018-05-22 - \textbf{Slutredovisning. Eller 23 maj}
    \item 2018-05-25 - \textbf{Egen utvärdering på blankett till handledaren}
    \item 2018-06-01 - \textbf{Slutrapport färdig och inlämnad}
    \item Identifierat problemområden
    \item Avslutat inledande inläsning.
    \item Skrivit klart handledningen.
    \item Första utkast till rapport.
    \item + andra deadlines
\end{itemize}


\newpage
\pagenumbering{gobble}

\bibliography{referenser} 
\addcontentsline{toc}{section}{Referenser}
\bibliographystyle{ieeetr}

\newpage
\section*{Bilagor}
\label{sec:bilaga}
\addcontentsline{toc}{section}{\nameref{sec:bilaga}}

\subsection*{Bilaga $\lambda$: Detaljerad lista över metod}

\subsubsection*{Inläsning}

\begin{itemize}
    \item Identifikation av problemområden.
        \begin{itemize}
            \item Kontakt med Åke och DNS. Studera kursutvärderingar.
            \item Reflektera över vad vi själva tyckt varit svåra områden då vi läst kursen.
            \item Se om Patrik har något kul att säga...
        \end{itemize}
    \item Studerande av existerande läromaterial, både inom ren fysik och liknande vårt material.
        \begin{itemize}
            \item Fysikboken.
            \item Åkes egna material.
            \item Boken \textit{Structure Interpretaton of Classical Mechanics}\cite{SICM}.
            \item Kursboken till kursen \textit{Matematikens domänspecifika språk}.
        \end{itemize}
    \item Existerande implementationer.
        \begin{itemize}
            \item OpenTA.
            \item Hamilton.
            \item MasteringPhysics.
            \item Fråga om Patrik kan något mer...
        \end{itemize}
    \item Tidigare forskning.
        \begin{itemize}
            \item Cezar och Patriks 2015 paper.
            \item 2016 års kandidatarbete.
            \item Artikeln \textit{DSL for the Uninitiated}.
            \item \textit{Communicating Mathematics: Useful Ideas from Computer Science}
        \end{itemize}
\end{itemize}

\subsubsection*{Implementation av domänspecifika språk}

Vid implementationen av ett/flera domänspecika språk behöver nedanstående punkter genomföras.

\begin{itemize}
    \item Hitta relevanta grundtyper inom fysik, exempelvis sträcka och massa.
    \item Hitta relevanta komposittyper, exempelvis hastighet och tryck.
    \item Utförligt typsystem.
    \item Dimensionskontroll.
    \item Modellera fysikens syntax i språket.
    \item Pedagogiska syntaxträd.
    \item Kombinatorer och konstruktorer.
    \item Hålla våra typer polymorfa.
\end{itemize}

\subsubsection*{Skrivande av läromaterial}

Vid skrivandet av läromaterialet kommer följande punkter ligga till grund.

\begin{itemize}
    \item Frågetecken: Gemensam vokabulär (som funkar när man pratar om både fysik och programmering, generics kontra polymorfism). Och som gör det möjligt att prata om dem i samma mening utan att byta språk och på så sätt brygga det semantiska gapet mellan områdena.
    \item Övningar
        \begin{itemize}
            \item Modellera ett fysikaliskt problem med vårt domänspecifika språk.
            \item Lös ett ''vanligt'' fysikaliskt problem med hjälp av vårt domänspecifika språk.
            \item Simuleringar i stil med \textit{Bouncing Balls}.
            \item Delar av fysiken vi inte behandlat lämnas som övning att själv implementera.
        \end{itemize}
    \item Gå igenom allmän teori (t.ex. Newtons lagar, krafter som verkar, etc) tillsammans med en parallell utveckling av ett domänspecifikt språk.
    \item Materialet ska vara enkelt att ta till sig.
    \item Verkligen exponera det DSL som vi gemensamt bygger för att påvisa kopplingen mellan fysik och programmering.
\end{itemize}

\end{document}








































