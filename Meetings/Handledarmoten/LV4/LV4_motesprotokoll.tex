%LaTeX-inställningar%%%%%%%%%%%%%%%%%%%%%%%%%%%%%%%%%%%%%%%%%%%%%%%%%
% Kompliera med pdflatex
\documentclass[a4paper, 10pt]{article}

\usepackage[utf8x]{inputenc}
\usepackage[swedish]{babel}
\usepackage{graphicx}

\usepackage{geometry} 
\geometry{a4paper} 
\geometry{margin=1in}
\setcounter{section}{1}
%%%%%%%%%%%%%%%%%%%%%%%%%%%%%%%%%%%%%%%%%%%%%%%%%%%%%%%%%%%%%%%%%%%%%%%

%Fyll i vid varje möte:
\newcommand{\tid}{11:30}
\newcommand{\plats}{4205}
\newcommand{\datum}{2018-02-09}

\newcommand{\sammankallande}{\textit{
Björn\\
Erik\\
Oskar\\
Johan\\
Handledare Patrik\\
}}

%Snabbkommandon
\newcommand{\sect}[1][]{\section*{\S \thesection. #1} \stepcounter{section}}
\newcommand{\para}{\paragraph \noindent}
\newcommand{\ssect}[1][]{\subsection*{#1}}

\begin{document}


%Rubrik etc
\section*{\center Mötesprotokoll - DSLsofMath LV4} 
\vspace{1em}
\textbf{Kl:} \tid , \datum \\
\textbf{Plats:} \plats \\

%Dagordning:
%Detta är en mall. Lägg till paragrafer och underparagrafer om det behövs.

\sect[Mötets öppnande]
\ssect[Närvarande]

\sect[Informationspunkter]

Nästa vecka är fackspråk innan mötet.

\sect[Statusrapport]

Halvtidsredovisningen. 
Saknas ett par saker om nuvarande status.

27de februari? Patrik undervisar. Går att byta?
Björn frågar det.

Gantt:
Vi gör dsl, ska börja med lärarmaterial.

\sect[Uppföljning och eventuella förändringar]

\sect[Kommande uppgifter]

Fackspråk handledning. 
Halvtidsredovisningen kompletterar vi senare när vi vet mer om nuvarande status om ett par veckor.

\sect[Övriga frågor]
Diskusion om DSL med Patrik.

Krav på bevisföring? Vad behöver eleverna kunna bevisa? Vi på data behöver inte bevisa förhållanden som på andra instutitioner?

Många av operationerna behöver vara dimensionslösa (ex sinus) ty det finns inte "sinus x meter".

Simplifiering och avrundningsfel? Ang att kolla om två saker är ekvivalenta. Syntaxträd skulle kunna "arrangeras om".

\sect[Nästa möte sker...]

11:30 Tisdag LV6. VI hoppar LV5 ty patrik ska på semester.

\sect[Mötets avslutande]


\end{document}

