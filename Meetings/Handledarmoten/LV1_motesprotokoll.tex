%LaTeX-inställningar%%%%%%%%%%%%%%%%%%%%%%%%%%%%%%%%%%%%%%%%%%%%%%%%%
% Kompliera med pdflatex
\documentclass[a4paper, 10pt]{article}

\usepackage[utf8x]{inputenc}
\usepackage[swedish]{babel}
\usepackage{graphicx}

\usepackage{enumitem}

\usepackage{geometry} 
\geometry{a4paper} 
\geometry{margin=1in}
\setcounter{section}{1}
%%%%%%%%%%%%%%%%%%%%%%%%%%%%%%%%%%%%%%%%%%%%%%%%%%%%%%%%%%%%%%%%%%%%%%%

%Fyll i:
\newcommand{\tid}{11:30}
\newcommand{\plats}{4205}
\newcommand{\lasvecka}{1}
\newcommand{\datum}{2018-01-19}
\newcommand{\present}{
Erik\\
Björn\\
Johan\\
Oskar\\
Handledare Patrik
}
\newcommand{\justerare}{Oskar}
\newcommand{\sekreterare}{Björn}
%Snabbkommandon
\newcommand{\sect}[1][]{\section*{\S \thesection. #1} \stepcounter{section}}
\newcommand{\para}{\paragraph \noindent}
\newcommand{\ssect}[1][]{\subsection*{#1}}

\begin{document}


%Rubrik etc
\section*{\center Protokoll Projektmöte LV\lasvecka} 
\vspace{1em}
\textbf{Tid:} \tid , \datum \\
\textbf{Plats:} \plats \\

%Protokoll:
%Observera att detta är ett exempel som speglar exemplet på kallelse. Protokollet skall utformas på så sätt att det varje vecka speglar den aktuella kallelsen. Paragrafer kan komma att läggas till eller justeras.
\sect[Mötets öppnande]
\ssect[Närvarande]
\present %Fyll i listan som startar på rad 20

\sect[Val av sekreterare och justerare]

\sekreterare \ väljs som sekreterare.\\
\\
\justerare \ väljs som justerare.

\sect[Godkännande av dagordning]

Ingen specifik dagordning var förberedd. Detta mötesprotokoll är en efterhandskonstruktion av det vi diskuterade.\\

Preliminärer.

Organiseringen av projektet.

Gruppkontrakt.

Planeringsrapport.

Bestämma ett projket.

Bokningar av handledningar.

\sect[Godkännande av föregående mötesprotokoll]
Första mötet med dagordning. Därav inget föregående mötesprotokoll.

\sect[Informationspunkter]
Daniel hoppade av kandidatarbetet. Nu är vi fyra kvar.

\sect[Val av ansvarspersoner]
Projektledare - Erik väljs\\
Dokumentations- och bokningsansvarig - Björn väljs\\
Verifikationsansvarig - Johan väljs\\
Gransknings- och presentationsansvarig - Oskar väljs


\sect[Statusrapport]
%Alla rapporterar individuellt status


\ssect[Individuellt]
Samtliga gruppmedlemmar har spenderat tid på att studera DSL.

\sect[Uppföljning och eventuella förändringar]
%Mötet diskuterar om milstolparna för denna veckan har nåtts. Om inte, hur ska de nås under kommande vecka?

\ssect[Val av projektarbete]
Vi har begränsat projektet till något fysikrelaterat. Ex.v att implementera ett DSL för kursen fysik för ingengörer (D2), med fokus på att det som lärarmaterial ska kunna stödja studenternas förståelse för fysik och programmering.

\sect[Kommande uppgifter]

\ssect[Projektet]
Planeringsrapporten.\\
Gruppkontrakt.\\
Bokningar.\\
Att gruppens medlemmar fortsätter bekanta sig med DSLer.\\
Bra att göra: Läsa föregående kandidatgrupps rapport.

%Vad är kommande uppgifter för varje individ?

\ssect[Individuellt]

Björn: Bokar handledningar. Fixa möteprotokoll och dagordningsmallar. Mailar om vi har förstått projektdagboken rätt.\\
Oskar: Granska mötesprotokollet. Lägger in presentationspunkter i planeringsrapporten.\\
Johan: Gör gantt prototyp i planeringsrapporten.\\
Erik: Skriva in eventuella extrapunkter i planeringsrapporten. Fixar slack.

\sect[Övriga frågor]
%Här tas frågorna som kom upp i §3 upp

\sect[Nästa möte sker...]

\ssect[Handledarmöte:]
%14:00-15:00 Onsdag 14/9.
LV2 fredag. 11:30 - 13:00.
Johan bokar.

\sect[Mötets avslutande]

\ \\
\parbox[b]{250pt}{
\parbox[b]{250pt}{
Vid protokollet
\ \\ \ \\ \ \\ 
\sekreterare \\
\_\_\_\_\_\_\_\_\_\_\_\_\_\_\_\_\_\_\_\_\_\_\_\_\_\_\_\_\_\_\_\_\_\_\_\_\_\_\_\_\_\_\_\_\_\_\_\_\_\_\_\_\_\_\_\_ \\
}
\\ \ \\
\parbox[b]{250pt}{
Justeras
\ \\ \ \\ \ \\ 
\justerare \\
\_\_\_\_\_\_\_\_\_\_\_\_\_\_\_\_\_\_\_\_\_\_\_\_\_\_\_\_\_\_\_\_\_\_\_\_\_\_\_\_\_\_\_\_\_\_\_\_\_\_\_\_\_\_\_\_ \\
}
}

\end{document}
