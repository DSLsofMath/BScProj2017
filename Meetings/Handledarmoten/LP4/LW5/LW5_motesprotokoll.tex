%LaTeX-inställningar%%%%%%%%%%%%%%%%%%%%%%%%%%%%%%%%%%%%%%%%%%%%%%%%%
% Kompliera med pdflatex
\documentclass[a4paper, 10pt]{article}

\usepackage[utf8x]{inputenc}
\usepackage[swedish]{babel}
\usepackage{graphicx}

\usepackage{enumitem}

\usepackage{geometry} 
\geometry{a4paper} 
\geometry{margin=1in}
\setcounter{section}{1}
%%%%%%%%%%%%%%%%%%%%%%%%%%%%%%%%%%%%%%%%%%%%%%%%%%%%%%%%%%%%%%%%%%%%%%%

%Fyll i:
\newcommand{\tid}{13:00}
\newcommand{\plats}{3511}
\newcommand{\lasvecka}{5}
\newcommand{\datum}{2018-04-25}
\newcommand{\present}{
Björn\\
Erik\\
Oskar\\
Handledare Patrik\\
}
\newcommand{\justerare}{Ingen}
\newcommand{\sekreterare}{Björn}
%Snabbkommandon
\newcommand{\sect}[1][]{\section*{\S \thesection. #1} \stepcounter{section}}
\newcommand{\para}{\paragraph \noindent}
\newcommand{\ssect}[1][]{\subsection*{#1}}

\begin{document}


%Rubrik etc
\section*{\center Mötesprotokoll LP4 LV\lasvecka} 
\vspace{1em}
\textbf{Tid:} \tid , \datum \\
\textbf{Plats:} \plats \\

%Protokoll:
%Observera att detta är ett exempel som speglar exemplet på kallelse. Protokollet skall utformas på så sätt att det varje vecka speglar den aktuella kallelsen. Paragrafer kan komma att läggas till eller justeras.
\sect[Mötets öppnande]
\ssect[Närvarande]
\present %Fyll i listan som startar på rad 20

\sect[Godkännande av dagordning]

\sect[Godkännande av föregående mötesprotokoll]

\sect[Informationspunkter]

LYAP blev testat i fredags av en person.

Vi har frågat grundskolegruppen om att göra en opponering med dom.

\sect[Statusrapport]
%Alla rapporterar individuellt status

\sect[Uppföljning och eventuella förändringar]
%Mötet diskuterar om milstolparna för denna veckan har nåtts. Om inte, hur ska de nås under kommande vecka?

\sect[Kommande uppgifter]

\sect[Övrigt]

Vi går igenom rapporten med Patrik.\\ 
\\
Vi lär behöva planera dry-runs för slutredovisning + opponering.\\
\\
Erik skickar till haskell-halvårs-brevet.\\
\\
Vi bör nämna DNS i rapporten.

\sect[Nästa möte sker...]

Möte onsdag 2a maj 13:15.

\sect[Mötets avslutande]

\end{document}
