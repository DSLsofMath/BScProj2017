%LaTeX-inställningar%%%%%%%%%%%%%%%%%%%%%%%%%%%%%%%%%%%%%%%%%%%%%%%%%
% Kompliera med pdflatex
\documentclass[a4paper, 10pt]{article}

\usepackage[utf8x]{inputenc}
\usepackage[swedish]{babel}
\usepackage{graphicx}

\usepackage{geometry} 
\geometry{a4paper} 
\geometry{margin=1in}
\setcounter{section}{1}
%%%%%%%%%%%%%%%%%%%%%%%%%%%%%%%%%%%%%%%%%%%%%%%%%%%%%%%%%%%%%%%%%%%%%%%

%Fyll i vid varje möte:
\newcommand{\tid}{11:30}
\newcommand{\plats}{4205}
\newcommand{\datum}{2018-02-02}

\newcommand{\sammankallande}{\textit{
Björn\\
Erik\\
Oskar\\
Johan\\
Handledare Patrik\\
}}

%Snabbkommandon
\newcommand{\sect}[1][]{\section*{\S \thesection. #1} \stepcounter{section}}
\newcommand{\para}{\paragraph \noindent}
\newcommand{\ssect}[1][]{\subsection*{#1}}

\begin{document}


%Rubrik etc
\section*{\center Dagordning - Grupp DSLsofMath LV3} 
\vspace{1em}
\textbf{Kl:} \tid , \datum \\
\textbf{Plats:} \plats \\

%Dagordning:
%Detta är en mall. Lägg till paragrafer och underparagrafer om det behövs.

\sect[Mötets öppnande]
\ssect[Närvarande]

\sect[Val av sekreterare och justerare]

\sect[Godkännande av dagordning]

Planeringsrapporten klar?
	Patriks kommentarer.

Kommande grejer?
	Vad vill vi starta med efter inläsning eller parallellt med inläsning.
	

\sect[Godkännande av föregående mötesprotokoll]

\sect[Informationspunkter]

Skrivmetodik - Dokumentera så mycket som möjligt av undersökningsarbetet och framtagandet av olika DSLs. Varför är en variant bättre eller sämre ex? Vi behöver mallar.

Vi har fått kritik från Andreas att vara tydligare i dagboken (antagligen den med TODOs och ansvarspersoner). Ett ökat antal timmar per vecka 

Björn o Oskar fixar halvtidsredovisningen
Erik o Johan fixar slutredovisningen

\sect[Statusrapport]

Vi har jobbat med planeringsrapporten samt inläsning.

Detaljer finns i loggboken.

\sect[Uppföljning och eventuella förändringar]

\sect[Kommande uppgifter]

Olika ansvarsområden för olika ämnesområden.
Ex derivator, kapitel? 

	Matematiken bakom fysiken.

	Derivator integraler

	Momentan(x), genomsnitts(x)

	Vektorer, vektorops
		Förklara med datatyper

	Newtons lagar

	Differentialkalkyl

	Dimensioner!

	Bra om vi presenterar exempel på fredagarna så att patrik kan ge respons.

	Figurer i fysik. Bollar på lutande plan osv.

\sect[Övriga frågor]

Kan man använda existerande verktyg på nätet för att presentera fysik + dsl enklare?
Haste för grafisk presentation i webbsidor. Kan göras liknande tryhaskell.org.

\sect[Nästa möte sker...]

\sect[Mötets avslutande]


\end{document}


