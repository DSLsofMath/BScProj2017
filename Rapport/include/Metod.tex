% Metoddelen redogör för vad du gjort och hur du gått tillväga; det är
% en beskrivning av den metod som ligger till grund för det du kommit
% fram till och hävdar i din rapport.

% Beskrivningen i metoddelen ska vara koncis snarare än helt
% uttömmande men ska samtidigt göra det möjligt att upprepa studien

% Metoddelen ska inte vara en omgjord labbinstruktion och den ska inte
% heller innehålla teori med mindre än att teoretiska hänsyn har haft
% en direkt inverkan på metoden.

% Metoddelen skrivs nästan alltid i dåtid (imperfekt) och ofta används
% passiv form för att beskriva forskningsaktiviteter.

\chapter{Metod/Genomförande}

% TODO: Är det intressant att ha med HUR vi, genom research, kom fram
% till att göra en specifik sak och varför den är bra?
%
% E.g. "Genom att läsa i bok A om område B så kom vi fram till att vi
% ska applicera metod C för att lösa problem D. Enligt bok A är metod
% C bra därför att ..."
% vs.
% "Vi valde att lösa problem D genom att applicera metod C. Metod C är
% bra därför att ...[ev. källhänvisning]."

% TODO: Nånting om hur processen såg ut, lite mer meta. I.e. hur
% arbetade vi i allmänhet? Scrum, vattenfall? Med issue-tracker, med
% jira? Versionshantering? Etc.

\begin{draft}

  Projektet har i grova drag genomförts i fyra överlappande
  faser. Först identifierades de områden inom mekanik som studenter
  verkar ha svårt för, eller som vi trodde skulle vara intressanta att
  arbeta med. Sedan skrevs både DSL och lärotext för dessa
  områden. Efter det anpassades texterna för att bli publicerbara som
  kapitel, och kompilerades till ett läromaterial i form av en
  HTML-bok som publicerades på en internethemsida. Slutligen gjordes
  en icke-rigorös evaluering av materialet, i form av att ett urval
  studenter fick läsa det och komma med feedback.

\end{draft}

% TODO: Sektion om våra litteraturstudier? Känns lite skumt att ha,
% men 2016 hade. Litteratur är väl bara intressant att ha med som
% refens när man refererar till fakta eller motiverar ett val? Själva
% studieprocessen, i.e. HUR man nådde beslutet, är väl inte alls lika
% intressant att ha med som VARFÖR man tog beslutet?

\begin{binge}
\section{Selektion av arbetsområden}

% TODO: Åke mailet hjälpte oss identifiera områden.
% TODO: Hjälpte det senare Åke mötet nåt? För just detta?

\begin{draft}
  Först kontaktades Åke Fäldt, som är examinator för kursen TIF085,
  Fysik för Ingenjörer. Åke befrågades om vilka områden han tycker att
  studenter verkar ha svårt för, och svarade att studenter i allmänhet
  verkar ha svårt för att sätta upp egna, mentala modeller för många
  problem och koncept. ``Man tar genvägar (som ofta är fel) och bygger
  inte från det som man är säker på gäller'' skrev han, och han pekade
  ut infinitesimalkalkyl som ett speciellt svårt område:
  ``Infinitesimalkalkyl är ju en av hörnpelarna i fysiken och
  studenterna har trots att de har läst ganska mycket matematik
  jättesvårt att använda detta på verkliga system''.

  För att identifiera mer specifika ämnesområden att arbeta med,
  studerades kursboken *Univeristy Physics*. Speciellt av intresse var
  kapitel som berörde saker som Åke Fäldt tidigare pekat ut som svåra,
  och kapitel som använde sig av specifik syntax. Domänspecifik syntax
  var av intresse att finna, då en betydlig del av domänspecifika språk
  är modellering av just syntaxen. Även områden som projektgruppen fann
  personligen intressanta, och områden som inte var av direkt intresse,
  men som utgjorde en kritisk beståndsdel av mer intressanta områden,
  valdes ut.

  De valda ämnesområdena delades upp och definierades sådana att de var
  fristående, i så god mån som möjligt, för att kunna arbetas med
  parallellt.

  De områden som valdes ut blev: Vektorer, Enheter, Momentan- och
  medel-rörelse, och Matematisk analys.

  Vektorer eftersom det är en viktig grundsten i mekanik. Alla krafter,
  hastigheter, och accelerationer, betraktas oftast som vektorer i
  planet eller rymden, och dessa är alla fundamental element i
  mekanik.

  Enheter eftersom det är viktigt för studenter att förstå sig på hur
  dimensioner påverkas av algebraiska operationer. Det kan också vara
  hjälpsamt att kunna utföra automatisk, datorassisterad
  dimensionsanalys på ens beräkningar.

  Momentan- och medel-rörelse eftersom de direkt utgör en stor delmängd
  av alla problem inom mekanik på den aktuella nivån. Väldigt många av
  de uppgifter studenter lär sig lösa inom mekanik är
  sträcka/hastighet/acceleration/kraft problem. Hur lång tid tar det att åka en
  sträcka om man har en viss medelhastighet? Om ett objekt med massa $m$
  påverkas av en kraft som varierar enligt $sin(t)$, vad är då
  momentanhastigheten vid $t=10$?

  Analys eftersom alla koncept i klassisk mekanik är relaterade genom
  matematisk analys. Mer specifikt används differenser för att beskriva
  medelrörelse, och infinitesimalkalkyl för att beskriva
  momentanrörelser. Vidare var infinitesimalkalkyl just det område som
  Åke Fäldt pekade ut som speciellt viktigt, och något som studenter har
  svårt för.
\end{draft}

% TODO: Nåt mer?

% TODO: Nåt om hur områdesvalen speglas i strukturen av
% läromaterialet?

\end{binge}

\begin{binge}
  \section{Implementation av DSL för områdena}

  % TODO: Hur? I haskell, med syntaxträd, etc.
  Först experimentering. Sen implementation i haskell med syntaxträd
  och sånt. Sen få flera dsl att fungera tillsammans. Sist
  problemlösning o sånt.

\subsection{Skriva DSL för ett område}

Varje gruppmedlem valde varsitt område att arbeta med. Man började med att
experimentera med DSL:et för att hitta bra sätt att representera området på,
vad som var tydligt och lätthanterat i datorn.

Det skedde också en del Haskell-inläsning av nya områden, exempelvis
typnivå-progammering, för att kunna göra DSL:er på bästa sätt.

När en tanke börjat formas så implementerades först DSL:et. När det till stora
delar var klart började förklarande brödtext skrivas till det, främst för att
förklara koden som skrivits.

När koden var färdig och kommenterad tillräckligt väl började brödtexten
uppdaters för att även innehålla mer kopplingar till fysik.

\subsection{Komposita områden}

Importera DSLerna för varandra för att göra mer komplicerade grejer.

Eller, \emph{Bruk av de mer fundamentala/teoretiska DSLerna för att
  angripa områden av mer ``tillämpad'' natur (såsom Krafter, Arbete,
  etc) )}(?)

!! Områden/moduler soom bygger vidare på redan implementerade områden.

stack, git kan säkert passa här för att beskriva hur vi samarbetade med sammanfogningen.

\section{Skriva lärotext till DSL}

\subsection{Didaktik/språk/utlärningsmetod}

Lättsamt språk o en gnutta humor för att hålla kvar
uppmärksamhet. Relaterat till Attention i ARCS modellen (2016 skrev om
det, såg vettigt ut).

\section{Skriva läromaterial för hur DSL appliceras för problemlösning}

Vari vi visar att DSLerna både är praktiskt användbara, likt Wolfram
Alpha, och att implementationen+applikationen hjälper oss förstå
mekanik i allmänhet och probleminstanserna i synnerhet.

% TODO: Motivera varför vi valde LHS
% TODO: Motivera varför vi valde hemsida istället för PDF.
% TODO: Motivera varför vi valde Markdown istället för LaTeX i LHS filerna.

\section{Sammanställning, presentation, och publicering}

Läromaterialet publiceras på en internethemsida, varpå man kan läsa
allt o ha skoj.

\subsection{Beskrivning}

Ett build-script hämtar .lhs källfilerna, i vilka brödtexten är
skriven med markdown. Rendrar med pandoc, och sätter in lite
navigationselement etc. med hjälp av eget templating-system. Manuellt
läggs sedan stoffet på gh-pages branchen för att automatiskt visas på
dslsofmath.github.io/BScProj2018.

Obs: Medan bygget är scriptat så är inte publiceringen det, och
ingenting genereras/publiceras automatiskt kontinuerligt. Måste köra
scriptet manuellt och lägga stoff på gh-pages branchen.

\subsection{Build-script}

I.e. implementation av python-build-scriptet i mer detalj.

Är detta ens intressant? Viktigt för att producera sidan såklart, men
inte intressant ur varken matte eller haskell/DSL perspektiv.

\subsection{Hemsidan}

Nåt om design, läslighet, grafik(?), navigation, avsiktligt undvikande
av javascript, etc.

Tänker lite samma med denna sektion som ovan. Har ju ingenting med
varken matte eller DSL att göra i sig, så kanske inte så intressant?
Samtidigt är det kanske lite intressant ur pedagogik-aspekten. Kan det
kanske vara lättare/roligare att lära sig om sidan är fin och
lättläst? Att javascript inte krävs gör att sidan kan visas ordentligt
även om man sitter i U-land med dålig/gammal/billig telefon.

\section{Test och återkoppling}

% Återkoppling från examinator (NAD): "Nils Anders Danielsson <nad@cse.gu.se>
% 27 Feb (1 day ago)
% to Patrik, Andreas
% Hi,Your BSc project groups both try to make tools for learning. I had some
% discussion with Andreas' group about their plans for evaluating how well
% their product works. My position is that, given the resource limits of
% these projects (and general problems of reproducibility in social
% sciences), it is very hard to perform an evaluation that gives useful
% results. I don't mind if your groups try to perform some kind of
% evaluation, but I suggest that you tell them to avoid overstating the
% importance of the evaluations in the final reports."
% Jag tror det är kompatibelt med det jag sagt tidigare - att göra en "ordentlig" utvärdering av det pedagogiska utfallet är komplicerat och tar (kalender-)tid.
% Informell utvärdering av en testgrupp bör dock ingå.

Test på försöksstudenter. Återkoppling med Åke(?).

Nog bra att vara explicit här med att det inte är en rigorös empirisk
studio, om inte det redan täckts väl i Avgränsningar.

\end{binge}
