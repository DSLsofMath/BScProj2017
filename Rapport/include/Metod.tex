% CREATED BY DAVID FRISK, 2016

\chapter{Metod}

\begin{binge}
Skapandet av läromaterialet har i grova drag haft tre faser. Först valdes olika
arbetsområden ut, som enskilt gick att arbeta med. Sedan Skapades läromaterial
för dessa områden. Till sist sammanfoagdes resultatet.

\section{Selektion av arbetsområden}

För att hitta områden att arbeta med studerades främst kursboken *Univeristy
Physics*. Där lästes de kapitel som ingick i *Fysik för ingenjörer*. Sådant som
verkade hade syntax som behövde förklaras, eller sådant som var svårt, eller
sådant som var spännande valdes ut. De områden som hittats sorterades upp i
grupper som var så fristående som möjligt för att kunna arbetas med på
parallellt.

De områden som valdes ut blev

- Vektorer

- Enheter

- Momentan och genomsnitt

- Differentalkalkyl

\section{Skapande av de områdena}

\subsection{Först områdena}

Varje gruppmedlem valde varsitt område att arbeta med. Man började med att
experimentera med DSL:et för att hitta bra sätt att representera området på,
vad som var tydligt och lätthanterat i datorn.

Det skedde också en del Haskell-inläsning av nya områden, exempelvis
typnivå-progammering, för att kunna göra DSL:er på bästa sätt.

När en tanke börjat formas så implementerades först DSL:et. När det till stora
delar var klart började förklarande brödtext skrivas till det, främst för att
förklara koden som skrivits.

När koden var färdig och kommenterad tillräckligt väl började brödtexten
uppdaters för att även innehålla mer kopplingar till fysik.

\subsection{Komposita områden}
!! Områden/moduler soom bygger vidare på redan implementerade områden.

Stack, git kan säkert passa här för att beskriva hur vi samarbetade med sammanfogningen.

\section{Hemsidan}

!! Hur skrevs den?
Kompilerades kontinuerligt.

\end{binge}
