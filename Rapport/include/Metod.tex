% CREATED BY DAVID FRISK, 2016
\chapter{Metod}
Som ett genomgående tema vill vi arbeta in återkoppling med Åke Fäldt och Patrik
Jansson så att vi vet att vi håller oss på banan och inte gör det för svårt för
oss själva eller potentiella studenter.

Den övergripande planen är att börja med att läsa in oss på fysik, domänspecika
språk och liknande projekt. Detta för att få en grundläggande uppfattning av hur
projeket kan tänkas se ut. Därefter kommer läromaterialet att skrivas. Eftersom
domänspecifika språk kommer presenteras invävt i läromaterialet kommer därför
också de domänspecika språken skapas parallellt med skrivandet. Vi vet ännu inte
hur vi bra kan fördela arbetet mellan gruppmedlemmarna. En möjlighet, beroende
på tidsbehovet, är att en/två skriver om mekanik och en/två om termodynamik
eller våglära. Under skrivandets gång kommer en del inläsning behöva göras
parallellt. Det kan handla både om fysik och domänspecika språk, till exempel
att jämföra vår implementation med likaratade implementationer.

För att hitta de områden datateknologer har problem med i \textit{Fysik för
ingenjörer} kommer vi prata med kursens föreläsare, med DNS
(Datateknologsektionen på Chalmers) samt reflektera över de delar vi själva
tyckte var svåra.

De primära källorna till inläsning av fysik kommer att vara kursboken
\textit{University Physics}\cite{UP} samt föreläsarens egna material i form av
anteckningar och övningsuppgifter. Till vår hjälp för att förstå hur man på ett
bra sätt kan skapa domänspecika språk inom fysik har vi boken \textit{Structure
and Interpretaton of Classical Mechanics}\cite{SICM}.

Se bilaga $\lambda$ för en mer detaljerad lista över hur vi tänkt lägga upp arbetet.
