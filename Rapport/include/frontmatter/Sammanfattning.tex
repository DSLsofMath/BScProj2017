
\thispagestyle{plain}			% Supress header

TODO: Perhaps include both Abstract + Sammanfattning on one page?

\begin{binge}

\section*{Sammandrag}

% Syfte med rapport + syfte med projekt
Denna rapport beskriver utvecklingen av läromaterialet ``Learn You a
Physics for Great Good''. Projektet utfördes som ett kandidatarbete vid institutionen
för Data- och informationsteknik på Chalmers tekniska högskola. Syftet
med projektet är att skapa ett läromaterial i form av en text som
presenterar fysik med hjälp av programmeringskonceptet
\textit{domänspecifika språk}, där de domänspecifika språken är
implementerade i programmeringsspråket Haskell. Vidare ska 
den pedagogiska nyttan av läromaterialet diskuteras samt hur väl fysik och
domänspecifika språk går att kombinera.

% Bakgrund, vår motivation
Bakgrunden till projektet är att den för studenter på Datateknik på
Chalmers obligatoriska fysikkursen \textit{Fysik för ingenjörer} har
haft ganska dåliga tentastatestik under flera år. Vi tror att en
faktor till att just datateknikstudenter får dåligt resultat i denna
kurs, är att studenterna finner ämnet irrelevant i förhållande till
resten av utbildningen. Vi tror också att detta problem kan lösas med ett
läromaterial som fungerar som en brygga mellan fysik och
programmering, och både visar på relevansen av ämnet och väcker
intresse för fysik. Ett ökat intresse för fysik leder förhoppningsvis
till bättre resultat i kursen.

% Resultat, hur läromaterialet blev
Det resulterande läromaterialet innehåller ett antal kapitel som
behandlar områdena: bevis, fysikalsika dimensioner, matematisk analys,
vektorer, partikelmekanik, och tillämpningar av dem. Varje kapitel
består av fungerande Haskell-kod tillsammans med beskrivande
text. Vissa kapitel bygger upp domänspecifika språk från grunden medan
andra kombinerar och tillämpar tidigare domänspecifika språk på
fysikaliska problem. Läromaterialet publicerades på en hemsida:
\url{https://dslsofmath.github.io/BScProj2018/} och dess källkod finns
fritt tillgänglig på:
\url{https://github.com/DSLsofMath/BScProj2018}. Rapporten beskriver
även de möten och diskussioner med utomstående som genomförts för att
förbättra utvecklingen, samt de informella tester som gjorts på det
färdiga läromaterialet.

% Diskussion/slutsats, vad vi kom fram till (mer än att bli färdiga med läromaterialet)
De slutsatser som nås i diskussionen av det utförda projektet är att
domänspecifika språk kan ha en pedagogisk nytta i
fysikundervisning. Eftersom de domänspecifika språken. Den strikta och
rigorösa naturen av Haskell och domänspecifika språk ger att den
processen av den fysikproblemlösningen de används för också blir
strikt och rigorös, utan möjlighet till intuitiva men felaktiga
genvägar. Detta tankesätt tror vi kan främja lärande om det förmedlas
till klassisk fysikundervisning.

% KEYWORDS (MAXIMUM 10 WORDS)
\vfill
Nyckelord: Domänspecifika språk, Klassisk mekanik, Fysikutbildning, Läromaterial, Funktionell programmering

\end{binge}

\newpage				% Create empty back of side
\thispagestyle{empty}
\mbox{}
