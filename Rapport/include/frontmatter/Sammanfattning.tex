
\thispagestyle{plain}			% Supress header

\section*{Sammanfattning}

% Syfte med rapport + syfte med projekt
Denna rapport beskriver utvecklingen av läromaterialet \textit{Learn You a
Physics for Great Good!}. Projektet utfördes som ett kandidatarbete vid institutionen
för Data- och informationsteknik på Chalmers tekniska högskola. Syftet
med projektet är att skapa ett textbaserat läromaterial som presenterar fysik med hjälp av programmeringskonceptet
\textit{domänspecifika språk}, där de domänspecifika språken är
implementerade i programmeringsspråket Haskell. Vidare ska 
den pedagogiska nyttan av ett läromaterial av detta slag diskuteras samt hur väl fysik och
domänspecifika språk går att kombinera.

% Bakgrund, vår motivation
Bakgrunden till projektet är att den för studenter på Datateknik (D) på
Chalmers obligatoriska fysikkursen \textit{Fysik för ingenjörer} har
haft ganska dålig tentastatistik i flera år. Vi tror att en
faktor till att just D-studenter får dåligt resultat i denna
kurs, är att studenterna finner ämnet irrelevant i förhållande till
resten av utbildningen. Vi tror också att detta problem kan lösas med ett
läromaterial som bryggar fysik och
programmering, och både visar på relevansen av ämnet och väcker
intresse för fysik. Ett ökat intresse för fysik leder förhoppningsvis
till bättre resultat i kursen.

% Resultat, hur läromaterialet blev
Det resulterande läromaterialet innehåller fem kapitel som
behandlar områdena fysikaliska dimensioner, matematisk analys,
vektorer och partikelmekanik, och tillämpningar av dem. Varje kapitel
består av fungerande Haskell-kod tillsammans med beskrivande
text. Vissa kapitel bygger upp domänspecifika språk från grunden medan
andra kombinerar och tillämpar tidigare domänspecifika språk på
fysikaliska problem. Läromaterialet publicerades på en hemsida\footnote{\url{https://dslsofmath.github.io/BScProj2018/}}
och dess källkod finns
fritt tillgänglig\footnote{\url{https://github.com/DSLsofMath/BScProj2018}}.

Rapporten beskriver
även de möten och diskussioner med som genomförts med utomstående. Syftet har var att
förbättra läromaterialet samt genomföra en informell utvärdering av det
färdiga läromaterialet.

% Diskussion/slutsats, vad vi kom fram till (mer än att bli färdiga med läromaterialet)
Slutsatserna är att
domänspecifika språk kan ha en pedagogisk nytta i
fysikundervisning. Den 
rigorösa naturen hos Haskell och domänspecifika språk gör att den
den fysikproblemlösningen de används för också blir
rigorös, utan möjlighet till intuitiva men felaktiga
genvägar. Detta tankesätt tror vi kan främja lärande om det förmedlas
till traditionell fysikundervisning.

% KEYWORDS (MAXIMUM 10 WORDS)
\vfill
Nyckelord: Domänspecifika språk, Klassisk mekanik, Fysikundervisning, Läromaterial, Funktionell programmering, Haskell

\newpage				% Create empty back of side
\thispagestyle{empty}
\mbox{}
