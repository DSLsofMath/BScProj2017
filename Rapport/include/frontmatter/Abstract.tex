%TODO: Make sure to define a macro for the headline once and for all. Then reuse it when needed.
\setlength{\parskip}{0.5cm}

\thispagestyle{plain}			% Supress header
\section*{Abstract}

% max 150 ord typ
\begin{binge}

%https://www.sfu.ca/~jcnesbit/HowToWriteAbstract.htm
% Punktlista på vad som ska vara med:
% 150 ord max
% whole thesis -> condenced form

% Snott från sammandrag:




% Building blocks:    
%   En mening om varje kapitel:
    Introduction (sv introduktion): Describes the starting point, goals and restrictions.
    Theory (sv. teori): Describes concepts of DSL, functional programming and learning models aimed for the learning material.
    Method (sv. genomförande): Describes the construction of the material, publishing, the procedure of testing with a testgroup and comments from keyfigures from the ordinary education.
    Results (sv. resultat): Describes the resulting material, the feedback from the testgroup and the comments from a physic teacher.
    Discussion (sv. diskussion): Reviews the methodology and the results. Also mentions possible further extensions and ethical dimensions of the material.
    Conclusions (sv. slutsatser): A final summary connecting the initial goals, with the methodology, and the result.


% KEYWORDS (MAXIMUM 10 WORDS)
\vfill
Keywords: Läromaterial, fysik, haskell, funktionell programmering.
\end{binge}
% Learning material, physics, haskell, functional programming.

\newpage				% Create empty back of side
\thispagestyle{empty}
\mbox{}
