\thispagestyle{plain}			% Supress header
\section*{Förord}

\begin{draft}

Denna rapport behandlar kandidatarbetet ``Matematikens domänspecifika språk'',
som genomfördes på Chalmers tekniska högskola under vårterminen 2018. Vi som har
utfört detta kandidatarbete är tre studenter från civilingengörsprogrammet
Datateknik vid Chalmers tekniska högskola och en student från det
datavetenskapliga programmet vid Göteborgs Universitet.

Vi vill tacka Patrik Jansson, vår handledare, som med sina kloka tankar och goda
råd agerat som ett fyrtorn när vi seglat på okända domänspecifika hav. Vi vill
tacka Åke Fäldt som tagit sig tiden att diskutera sin egen kurs, vårt
läromaterial och gett oss tips och råd under utvecklingen. Vi vill tacka de
testare, både individer och grupper, som tagit sig tiden att studera och läsa
igenom vårt, ibland halvfärdiga, material och gett oss den kritik vi behövde för
att sporras till vidareutveckling. Vi vill tacka Jeff Chen vars tankar och idéer
om potentiella vidareutvecklingar för projektet gav oss ett helt nytt perspektiv
under arbetets gång.

Slutgiltligen vill vi tacka Miran Lipovača vars hemsida ``Learn You a
Haskell for Great Good!'' har både inspirerat utformningen av vår hemsida och
agerat som ett läromaterial för våra egna inledande studier av det fantastiska
programmeringsspråket Haskell. 

\end{draft}

\vspace{1.5cm}
\hfill
Författarna, Göteborg, Maj 2018.

\newpage				% Create empty back of side
\thispagestyle{empty}
\mbox{}
