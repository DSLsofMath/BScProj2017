
\chapter*{Ordlista}
\markboth{Ordlista}{}

\textbf{Datastudent} En student med bakgrund inom data och IT.

\textbf{Dimensioner} Syftar på \textit{fysikaliska} dimensioner.

\textbf{DSL} Förkortning av \textit{Domain Specific Language}, engelska för domänspecifikt språk.

\textbf{Fysikläraren} Åke Fäldt.

\textbf{Fysik för ingenjörer} En fysikkurs som är obligatorisk för studenter på civilingenjörsprogrammet Datateknik på Chalmers. Den ges i årskurs 2 och innehåller grunderna i klassisk mekanik, termodynamik och vågrörelselära.

\textbf{Grundläggande område} Ett område i läromaterialet som inte bygger på något annat område. Till exempel vektorer.

\textbf{Komposit område} Ett område i läromaterialet som bygger på andra områden. Till exempel lutande plan, som använder sig av vektorer.

\textbf{Literate Haskell} Litterat programmering i Haskell. Se avsnitt \ref{sec:lhs}

\textbf{Läromaterialet} Syftar på den pedagogiska text som projektet resulterat i.

\textbf{Semantik} Vad meningar (skapade i en syntax) har för betydelse. Se avsnitt \ref{sec:syntax}.

\textbf{Syntax} Grammatiken för ett språk, som beskriver hur meningar konstrueras i det. Se avsnitt \ref{sec:syntax}.

\textbf{Syntaxträd} En trädrepresentation av syntax. Se avsnitt \ref{sec:syntax}

\textbf{University Physics} Den fysikbok som används i Fysik för ingenjörer.

\textbf{Åke Fäldt} Föreläsare och examinator i kursen Fysik för ingenjörer.
