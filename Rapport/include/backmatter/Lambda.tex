\chapter{Bilaga $\lambda$}

\section*{Inläsning}

\begin{itemize}
    \item Identifikation av problemområden.
        \begin{itemize}
            \item Kontakt med Åke Fäldt och DNS. Studera kursutvärderingar.
            \item Reflektera över vad vi själva tyckt varit svåra områden då vi
          läst kursen.  \end{itemize}
    \item Studerande av existerande läromaterial, både inom ren fysik och liknande vårt material.
        \begin{itemize}
            \item Fysikboken.
            \item Åke Fäldts egna material.
            \item Boken \textit{Structure Interpretaton of Classical Mechanics}\cite{SICM}.
            \item Kursboken till kursen \textit{Matematikens domänspecifika språk}.
        \end{itemize}
    \item Existerande implementationer.
        \begin{itemize}
            \item OpenTA.
            \item Hamilton.
            \item MasteringPhysics.
        \end{itemize}
    \item Tidigare forskning.
        \begin{itemize}
            \item Cezar och Patriks 2015 forskningsartikel.
            \item 2016 års kandidatarbete.
            \item Artikeln \textit{DSL for the Uninitiated}.
            \item \textit{Communicating Mathematics: Useful Ideas from Computer Science}
        \end{itemize}
\end{itemize}

\section*{Implementation av domänspecifika språk}

Vid implementationen av ett/flera domänspecika språk behöver nedanstående punkter genomföras.

\begin{itemize}
    \item Hitta relevanta grundtyper inom fysik, exempelvis sträcka och massa.
    \item Hitta relevanta komposittyper, exempelvis hastighet och tryck.
    \item Utförligt typsystem.
    \item Dimensionskontroll.
    \item Modellera fysikens syntax i språket.
    \item Pedagogiska syntaxträd.
    \item Kombinatorer och konstruktorer.
    \item Hålla våra typer polymorfa.
\end{itemize}

\section*{Skrivande av läromaterial}

Vid skrivandet av läromaterialet kommer följande punkter ligga till grund.

\begin{itemize}
    \item En gemensam vokabulär som fungerar när man skriver om både
      fysik och programmering (generics kontra polymorfism), och som gör det
      möjligt att prata om dem i samma mening utan att byta språk och på så sätt
      brygga det semantiska gapet mellan områdena.  
    \item Övningar
        \begin{itemize}
            \item Modellera ett fysikaliskt problem med vårt domänspecifika språk.
            \item Lös ett ''vanligt'' fysikaliskt problem med hjälp av vårt domänspecifika språk.
            \item Simuleringar i stil med \textit{Bouncing Balls}.
            \item Delar av fysiken vi inte behandlat lämnas som övning att själv implementera.
        \end{itemize}
    \item Gå igenom allmän teori (t.ex. Newtons lagar, krafter som verkar, etc) tillsammans med en parallell utveckling av ett domänspecifikt språk.
    \item Materialet ska vara enkelt att ta till sig.
    \item Verkligen exponera det DSL som vi gemensamt bygger för att påvisa kopplingen mellan fysik och programmering.
\end{itemize}
