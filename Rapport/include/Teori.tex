% CREATED BY DAVID FRISK, 2016

\chapter{Teori}

\begin{binge}
Ska vi som de gjorde tidigare, förklara Git, Haskell, DSL, och fysik?

\section{Haskell och funktionell programmering}

VARFÖR VALDE VI HASKELL??

I funktionell progammering uttrycker man allting i små och självständiga
funktioner. Rekursion används ofta. Fördelen med detta är att programmen blir
koncisa och de tenderar sakna de "progammeringsteknsika" delar som behövs för
att få progammet att fungera, men som inte tillför någon betydelse till det man
uttrycker.

Haskell är ett funktionellt progammeringsspråk med ett starkt typsystem.
VARFÖR?

\section{Domänspecifika språk}

Ett domänspecifikt språk är som namnet låter ett språk som är gjort till en
viss domän. En domän kan vara t.ex. fysikaliska enhter, eller tillgångar och
skulder i ett företag. Eftersom språket är specifikt för domänen kan saker i
den uttryckas enklare än i ett generellt språk. VARFÖR BLIR DET ENKLARE? 


LÄNGRE FÖRKLARING??
Det finns två kategorier av domänspecifika språk, fristående och inbäddade.
Skillnaden är att ett fristående är ett nytt progammeringsspråk från grunden
medan ett inbäddat är skapat i ett värdspråk, och använder det språkets syntax.
I detta fall kommer inbäddade domänspecika språk att skapas i Haskell (Kanske
onödigt att säga).

\section{Syntaxträd och deras evaluering}

Ett typexempel på ett domänspecifkt språk är ett syntaxträd för algebraisk uttryck, här kodat i Haskell.

    data Expr = Expr :+: Expr
                Expr :*: Expr
                Const Double
                VarX

    exempel = (Const 7.0 :+: VarX) :*: ((VarX :+: Const 10.0) :*: VarX)

Ha med en bild på detta.

!!!Plus förklaring om evaluering!

\section{Fysik vi behandlar och Fysik för ingenjörer}

*Fysik för ingenjörer* är en fysikkurs som är obligatorisk för Datateknik i 2:an. Det är en grundläggande fysikkurs som behandlar mekanik, termodynamik och vågrörelselära

Den innehåller även en hel del tillämpad matematik, exempelvis vektorer och differentialkalkyl. Det användas bland annat vid beräkning av värmeledning.

Vi har utgått från kursplanen när vi plockat ut våra områden...

\section{LHS, Pandoc, HTML-hemsidan}

Vad är LHS??

Vad är Pandoc??

Hur generar vi hemsidan???

build.py????

\end{binge}
