% CREATED BY DAVID FRISK, 2016

\chapter{Teori}

\section{Domänspecifika språk}
\begin{draft}
  Ett domänspecifikt programmeringsspråk är ett språk som är avgränsat till ett
  specifikt domän. Detta domän kan ta många former, det kan vara ett språk för
  att formatera text på en hemsida (HTML), det kan vara ett språk för att
  interagera med en databas (SQL), ett språk för ett beskriva hur karaktärer ser
  ut (typsnitt). Användningsområden för dessa språk är väldigt begränsade men
  denna begränsning gör det möjligt att utveckla ett rikt och lättanvändligt
  språk för just detta område. 

  Motsatsen till ett domänspecifikt språk är ett generellt språk (C, Java,
  Python) som är turingkompletta, vilket betyder att det kan uttrycka alla
  beräkningsbara problem i dem och även lösa dem givet tillräckligt med tid och
  minnestillgångar. Begränsningen med dessa generella språk är just deras egen
  generaliserbarhet, eftersom de har stöd för alla typer av beräkningar så blir
  både läsbarheten och användarvänligheten lidande.
  
  Ett domänspecifikt språk kan antingen implementeras som ett fristående språk
  eller bäddas in i ett redan existerande språk. De domänspecifika språk som
  utvecklats inom detta projekt är inbäddade i språket \textit{Haskell}.
\end{draft}

\begin{binge}
VARFÖR BLIR DET ENKLARE? 
\end{binge}

\section{Haskell och funktionell programmering}

\begin{draft}
  Haskell är ett funktionellt programmeringsspråk som lämpar sig bra för att
  implementera ett domänspecifikt språk i. Anledningen till detta är den lätthet
  som man kan skapa nya datatyper och klasser för att representera grundstenarna
  i det nya språk, och även dess mönstermatchning som gör det möjligt att på
  enkelt sätt bryta isär komplexa datatyper för evaluering.
\end{draft}

\section{Syntaxträd och deras evaluering}

\textbf{???Osäker på om detta avsnitt är nödvändigt???}

\begin{binge}

Ett typexempel på ett domänspecifkt språk är ett syntaxträd för algebraisk uttryck, här kodat i Haskell.

    data Expr = Expr :+: Expr
                Expr :*: Expr
                Const Double
                VarX

    exempel = (Const 7.0 :+: VarX) :*: ((VarX :+: Const 10.0) :*: VarX)

Ha med en bild på detta.

!!!Plus förklaring om evaluering!
\end{binge}

\section{Fysik vi behandlar och Fysik för ingenjörer}

\begin{draft}
  \emph{Fysik för ingenjörer} är obligatorisk kurs för studenterna på det
  datatekniska programmet. Kursen täcker grundläggande fysikområden såsom
  mekanik, termodynamik och vågrörelselära. Det ingår även en del tillämpad
  matematik såsom vektorer och differentialkalkyl. Dessa områden ämnas att
  täckas in av de domänspecifika språken och det tillhörande läromaterialen.
\end{draft}

\section{Hemsidan}

\begin{draft}
  Programfilerna har skrivits i ett språk som kallas \textit{literate haskell}
  som kombinerar vanlig Haskell-kod med brödtext till något som kan tolkas av en
  kompilator som två separata saker. \textit{Pandoc} är ett program som används
  för att generera html-filer med läromaterialet som är den slutgiltiga produkt
  som finns tillgänglig på hemsidan.
\end{draft}
