
\chapter{Slutsatser}

\begin{binge}

Projektets mål var att konstruera ett läromaterial som modellerar fysik med hjälp av domänspecifika språk samt diskutera hur bra det går och om det finns en pedagogisk nytta i det. Bakgrunden låg i att projektgruppen ville väcka intresse för fysik hos datastudenter genom att presentera det ur ett funktionellt programmeringsperspektiv. Det skulle förhoppningsvis kunna förbättra den mindre bra tentastatistiken i kursen Fysik för ingenjörer.

Resultatet blev ett läromaterial bestående av domänspecifika språk i Haskell sammanvävt med lärotext som förklarar dem. Det innehåller kapitel för dimensioner, matematisk analys, partikelmekanik, vektorer och tillämpningar av dem på fysikaliska problem. Det ingår även programmeringsövningar.

För att undersöka den pedagogiska nyttan hölls en ytterst kort utvärdering med en testgrupp samt två möten med Åke Fäldt, föreläsare och examinator för Fysik för ingenjörer. Det framgick att det möjligtvis fanns en nytta med att presentera fysik på det rigorösa sätt som görs med hjälp av domänspecifika språk. Men för att verkligen vara säker på det krävs en nogrann undersökning huruvida studenter blir bättre på fysik med hjälp av ett läromaterial av detta slag. Det är detta som är den stora frågan och som definitivt behöver undersökas närmare.

\end{binge}
