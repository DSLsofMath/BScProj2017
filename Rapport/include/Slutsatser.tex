
\chapter{Slutsatser}

Projektets mål var att konstruera ett läromaterial som modellerar fysik med
hjälp av domänspecifika språk samt att diskutera hur väl det fungerar och om det finns en
pedagogisk nytta i det. Bakgrunden låg i att projektgruppen ville väcka intresse
för fysik hos datastudenter genom att presentera det ur ett
programmeringsperspektiv. Det skulle förhoppningsvis kunna förbättra den mindre
bra tentastatistiken i kursen Fysik för ingenjörer.

Resultatet blev ett läromaterial bestående av domänspecifika språk i Haskell
sammanvävt med en lärotext som förklarar dem. Det innehåller kapitel för
dimensioner, matematisk analys, partikelmekanik, vektorer och tillämpningar av
dem på fysikaliska problem. Det ingår även programmeringsövningar.
Läromaterialet publicerades på en hemsida som är fritt tillgänglig för alla att
besöka.

För att undersöka den pedagogiska nyttan hölls två möten med Åke Fäldt, föreläsare och examinator för Fysik
för ingenjörer. Det framgick från mötena med Fäldt att det fanns en klar
pedagogisk nytta med det tillvägagångssätt som vi använt för att beskriva fysik.
Den rigorösa struktur som Haskell tvingar en att använda ger inte utrymme för en
student att hoppa över delar av förståelsen för ett fysikproblem. Istället
tvingas studenten att implementera hela lösningen från grunden och detta kan
ge en djupare förståelse för problemet och fysiken i stort.

Det faktiska resultatet från detta tillvägagångssätt, läromaterialet,
testades av en testgrupp. Från denna grupp fick vi mycket positiv kritik och de
tyckte att läromaterialet var pedagogiskt, roligt och intressant. Men för att
verkligen kunna dra några slutsatser från detta krävs en mycket mer noggrann
undersökning huruvida studenter blir bättre på fysik med hjälp av ett
läromaterial av detta slag.

Till sist vill vi säga vi anser att det material vi har producerat har följt
våra mål väl eftersom det är roligt, lättläst och intressant. Och även om den
pedagogiska nyttan av materialet inte är ordentligt testat tycker vi alla i
projektgruppen att vi har dragit stor nytta av att utveckla det.
