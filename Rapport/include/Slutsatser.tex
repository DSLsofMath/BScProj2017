
\chapter{Slutsatser}

\begin{draft}

Projektets mål var att konstruera ett läromaterial som modellerar fysik med
hjälp av domänspecifika språk samt diskutera hur  det går och om det finns en
pedagogisk nytta i det. Bakgrunden låg i att projektgruppen ville väcka intresse
för fysik hos datastudenter genom att presentera det ur ett funktionellt
programmeringsperspektiv. Det skulle förhoppningsvis kunna förbättra den mindre
bra tentastatistiken i kursen Fysik för ingenjörer.

Resultatet blev ett läromaterial bestående av domänspecifika språk i Haskell
sammanvävt med en lärotext som förklarar dem. Det innehåller kapitel för
dimensioner, matematisk analys, partikelmekanik, vektorer och tillämpningar av
dem på fysikaliska problem. Det ingår även programmeringsövningar.
Läromaterialet publicerades på en hemsida som är fritt tilgänglig för alla att
besöka.

För att undersöka den pedagogiska nyttan hölls en ytterst kort utvärdering med
en testgrupp samt två möten med Åke Fäldt, föreläsare och examinator för Fysik
för ingenjörer. Det framgick från mötena med Fäldt att det fanns en klar
pedagogisk nytta med det tillvägagångssätt som vi använt för att beskriva fysik.
Den rigorösa struktur som Haskell tvingar en att använda ger inte utrymme för en
student att hoppa över delar av förståelsen för ett fysikproblem, istället
tvingas studenten att implementera hela lösningen från grunden och detta kan
ge en djupare förståelse för problemet och fysiken i stort. 

Det faktiska resultatet från detta tillvägagångssätt, läromaterialet,
testades av en testgrupp. Från denna grupp fick vi mycket positiv kritik och de
tyckte att läromaterialet var pedagogiskt, roligt och intressant. Men för att
verkligen kunna dra några slutsatser från detta krävs en mycket mer nogrann
undersökning huruvida studenter blir bättre på fysik med hjälp av ett
läromaterial av detta slag. Det är detta som är den stora frågan och som
definitivt behöver undersökas närmare.

\end{draft}
