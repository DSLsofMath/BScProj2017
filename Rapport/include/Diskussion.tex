% CREATED BY DAVID FRISK, 2016

\chapter{Diskussion}

\begin{binge}
\section{Tillvägagångssättet av skapandet}

Tanke: I metod/teori varför vi valde att just Haskell o.s.v. Beskrivning av vårt iterativa arbetssätt, hur vi sökte i mörkret. Här i diskussion kanske mer varför det var nödvändigt...

Annan tanke: antingen separat resultat-kapitel för "misslyckade" försök, som DSL för lutande plan, eller att vi har det här..

Vi har kanske inte varit så strukturerade, utan bara valt ut något område på måfå som vi kände för, t.ex. valet av termodynmik.

Blev så för svårt att veta vad som DSL lämpligt för.

Största svårigheten var att välja områden som skulla passa bra för DSL. Som att
famla i mörkret.

Kan skriva om misslyckade försök att skapa DSLer för vissa saker. T.ex. lutande plan (DSL för det, ej problemlösning av det området med andra DSL:er), bevis med Haskells typsystem...

\section{Domänspecifika språk och fysik}

\subsection{Vad för slags områden är DSLs lämpligt att göra för?}

Analys känns som ett område där DSLs kan vara lämpligt. Likaså dimensioner och
vektorer. Vad har dessa gemensamt?

Gemensamt för de ovanstående är att de är matematiska, har en fix struktur,
rigoröst hur de fungerar och har "data och operationer".

Orkar inte göra Latex tabell...
| DSL / data         | Ex operationer               |
|--------------------|------------------------------|
| Dimensioner        | Multiplikation, division     |
| Vektorer           | Skalärprodukt, vektorprodukt |
| Analys, funktioner | Derivera, multiplicera       |

Operationerna på datan resultar också i ny data av samma slag (t.ex.
vektorprodukt av två vektorer ger en ny vektor)

Varför gör detta dem lämpliga att göra DSLs för? För datan har en fix form. Och
operaion av data blir ny data som evalueras till någon av formerna.

Detta gör DSL till områdena lämpliga ur två synvinklar. Dels enkelt att göra
det rent praktiskt i Haskell. Dels för att man med DSL:et strukuretat upp och
tydliggjort all data och operationer som går att göra.

Skillnad på enkelt att göra rent tekniskt, och vad som är lämpliga DSLs för att
underlätta förståelsen av något.

\newpage

DSLs var svårare för lutande plan och termodynamik (och problemlösning i
allmänhet). Vad har dessa gemensamt?

Gemensamt för dessa är att det finns teoretiska samband/ekvationer som
relaterar olika egenskaer i problemet. T.ex. för det lutande planet `a = g *
sin v` är sambandet mellan `a` och `v` (och `g` som dock är en konstant bara).
För termodynamik är det att t.ex. samband mellan inre energi, antal atomer i
gasen och temperaturen.

Visst kan man modellera dessa samband. Men vad för nytta gör det?
Problemlösning handlar om att känna till vilka samband som finns och tillämpa
dem på olika sätt beroende på uppgift.

Man kan ju programmera en ekvationslösare, men den måste vara väldigt mekanisk
av sig. Gör det kanske inte enklare för en själv att lösa problem.

Och vad är skillnaden mellan dessa två kategorier av områden?

Den viktiga skillnaden är att t.ex. analys har tydlig data och operationer
medan problemlösning som lutande plan har ett gäng samband som man använder
beroende på behov.

\subsection{Gör DSLs så att fysik blir enklare att förstå?}

Ex med lutande plan: man kan betrakta ett sådant problem som en samling
ekvationer. Man har några kända värden och med hjälp av ekvationerna ska man
hitta den sökta obekanta. Hjälper verkligen ett DSL till att man blir bättre på
detta?

Med enheter, räcker det inte att förklara skillnaden mellan storheter,
dimensioner och enheter på ett så grundligt sätt vi gjorde, utan att blanda in
DSLs? Problemet i fysik för Data kanske är att det inte förklras grundligt och
mycket är underförsått. Behöver man ens förstå enheter grundligare för att
klara fysik bättre?

Man kan också se det som att DSLs och denna extrakunskap vi presenterat är för
att göra fysik intressant genom att visa på vad för kopplingar till
programmering man kan göra, även om områdena vi behandlat inte är direkt de
områden som man behöver förstå för att klara kursen. Projektet kan ses som ren
kuriosa som kan vara intressant.

\end{binge}
