% CREATED BY DAVID FRISK, 2016

\chapter{Introduktion}

\section{Bakgrund}

\begin{draft}

På civilingenjörsprogrammet Datateknik på Chalmers ingår den obligatoriska fysikkursen
\textit{Fysik för ingenjörer}. Tentastatistiken för denna kurs är
inte jättebra\cite{tentastatistik}. Vi
tror att många studenter på datateknik (``datateknologer'') finner denna kurs
svår eller ointressant, och att detta leder till att ungefär en
tredjedel av kursdeltagarna får underkänt på tentamen.

Examinatorn för kursen ``Fysik för ingenjörer, TIF085 (2016)'' Åke Fäldt, tycker
att studenter i allmänhet verkar ha svårt för att sätta upp egna
modeller. De baserar sina mentala modeller helt eller delvis på
intuition och felaktiga antaganden, istället för definitioner och
bevisade satser som man är säker på gäller. Detta leder till att de
tar genvägar som ofta är fel.

Detta tror projektgruppen skulle kunna lösas med avstamp från kursen ``Domain Specific Languages of
Mathematics'' (``DSLsofMath'') eller ``Matematikens domänspecifika språk''
vilket är en valbar kurs på kandidatnivå för studenter på Chalmers och
Göteborgs Universitet.

\iffalse
Domänspecifika språk, kan
förklaras som ett språk konstruerat för ett specifikt område, d.v.s. en
domän. Språket kan användas för att enklare uttrycka uttryck inom
domänen, till exempel Newtons andra lag $F=m \cdot a$, än vad som är
möjligt inom generella (programmerings) språk. Exemplet ovan kan i ett 
domänspecifikt språk evalueras enklare m.h.a. ett syntaxträd och mönstermatchning,
gentemot ett generellt språk där exempelvis en rekursiv swith-case sats skulle kunna användas. 
Dock med mer overhead\todo{Visa bild på syntaxträd för m * a}. 
\fi

Konkret så presenterar kursen matematik så som derivator, komplexa
tal och matriser ur ett funktionellt programmeringsperspektiv i det funktionella programmeringsspråket
Haskell. Dessa för studenterna bekanta verktyg används för att lösa
matematiska problem så som modellering av syntax, evaluering till
semantiska värden och datorassisterad bevisföring.


För vidare läsning
rekommenderas \textit{DSL for the Uninitiated}.\cite{DSLU} Projektgruppen ämnar att implementera 
domänspecifika språk för att beskriva fysik ur en alternativ vinkel.\todo{Ska nämnas en gång tidigt och en gång sammanfattningsvist nedan, eller beskriver vi vad vi ska göra längst upp innan bakgrund?}

I kursen DSLsofMath, var år 2016 var Cezar Ionescu huvudföreläsare, och från 2017 är Patrik Jansson
huvudföreläsare, vilka har beskrivit avseendet med kursen genom en artikel \cite{tfpie2015}. Det direkta målet 
med kursen DSLSofMath är att förbättra den matematiska utbildningen för datavetare och den
datavetenskapliga utbildningen för matematiker, där den grundläggande idéen bakom kursen är: 

\begin{center}
  ``[\dots] att uppmuntra studenterna att närma sig matematiska domäner från ett
  funktionellt programmeringsperspektiv: att ge beräkningsbevis (calculational
  proofs); att vara uppmärksamma på syntaxen för matematiska uttryck; och,
  slutligen, att organisera de resulterande funktionerna och typerna i
  domänspecifika språk.''\cite{lecture-notes}\cite{tfpie2015} 
\end{center}

Vi vill alltså med hjälp av domänspecifika språk som vi implementerar presentera fysik ur ett alternativt perspektiv, likt
på det sättet DSLsofMath presenterar kopplingar mellan matematik och programmering. Förhoppningen är att detta ska göra kopplingen mellan datateknikprogrammet och fysik tydligare och därmed underlätta lärandet.


En analogi:

Studenterna hade svårt för matte $\rightarrow $ DSLsofMath.\\
Studenterna har svårt för fysik $\rightarrow $ Learn you a physics.


Angående tidigare forskning och studier, finns även på MIT har en kurs inte helt olik DSLsofMath tidigare givits som berör både
fysik och domänspecifika språk (''DSL'').\ ``Classical Mechanics: A Computational Approach'' gavs av
Prof.\ Gerald Sussman och Prof.\ Jack Wisdom bl.a. år
2008.\cite{classical-mechanics-course-mit-2008}
Denna kurs på avancerad nivå studerar de fundamentala principerna av klassisk
mekanik med hjälp av beräkningsidéer för att precist formulera principerna av
mekanik, med början i Lagranges ekvationer och avslut i
perturbationsteori (teori för approximationer av matematiska lösningar). I kursen används boken ``Structure and %nämna referens till boken?
Interpretation of Classical Mechanics'' av Sussman, Wisdom och Mayer,
vilken förklarar fysikaliska fenomen genom att visa datorprogram för att
simulera dem, skrivna i språket Scheme.\cite{SICM}. Denna typen av kurser ter sig ovanliga, och är, till våran kännedom, den enda kursen bortsett från DSLsofMath på Chalmers som knyter samman matematik, fysik och programmering.

Utöver DSLsofMath-kursen har det även tidigare gjorts ett kandidatarbete om DSL 
här på Chalmers. Vårterminen 2016 utfördes kandidatarbetet
``Programmering som undervisningsverktyg för Transformer, signaler och
system. Utvecklingen av läromaterialet TSS med DSL'' av fem studenter
från Datateknik och Teknisk Matematik på Chalmers. Arbetet bestod av
utveckling av läromaterial med tillhörande programmeringskod,
uppgifter och lösningar, som komplement till existerande kurser i
signallära.\cite{kandidat2016}
\end{draft}

% Flytta en del från bakgrund till syfte? 
% 

\section{Projektets mål}

\begin{draft}

% Annan formulering från nån annanstans
\iffalse
Målet med projektet är att skapa domänspecifika språk för fysik samt ett   
tillhörande läromaterial. Läromaterialets syfte är att beskriva fysiken och
dess koppling till de domänspecifika språken projektgruppen utvecklat.

Nedan beskrivs hur att vi vill väcka intresse för fysik hos
datateknologer genom att presentera fysik ur ett annat perspektiv.
\fi


%NYA

Tanken med detta kandidatarbete är att angripa fysik på ett sådant sätt att ämnet blir både roligt och intressant för datastudenter, och därmed förhoppningsvis också enklare. Detta är likt premissen bakom kursen DSLsofMath och kandidatarbetet från 2016, som istället för fysik behandlade matematik respektive signallära.

Mer konkret ska ovanstående göras genom att skapa ett läromaterial. Läromaterialet ska bestå av domänspecifika språk, skrivna i Haskell, som modellerar fysik sammanvävt med förklarande brödtext. Läromaterialet ska i slutändan publiceras på en hemsida och all källkod ska finnas öppet tillgänglig.

En del av målet är också att ta reda på hur en kombination av domänspecifika språk och fysik kan se ut, samt om det finns en pedagogisk nytta i att kombinera dem.

\iffalse
GAMLA

Vår tanke med detta kandidatarbete är att, likt premissen bakom kursen DSLsofMath
och kandidatarbetet från 2016, angripa fysik på ett sådant sätt att ämnet blir både
intressant och roligt för datateknologer, och därmed förhoppningvis
också enklare. Med hjälp av domänspecifika språk skrivna i Haskell för att
modellera fysik, d.v.s.\ ett programmeringspråk som används på flera
datakurser, tror vi att kopplingen mellan fysikkursen och
datateknikprogrammet kan göras tydligare och lärandet kan underlättas.
Förhoppningen är att bl.a.\ det kraftfulla typsystemet i Haskell ska
hjälpa studenter att bygga mentala modeller som är korrekta och inte
bygger på felaktig intuition eller felaktiga antaganden.

Läromaterialet är menat att i slutändan bestå av en hemsida. Källkoden för projektet
skall finnas offentligt tillgänglig.

Att ta reda på hur en kombination mellan fysik och domänspecifika språk kan se ut, samt om det finns en pedagogisk nytta.   %Om vi sätter det som mål måste vi underbygga det med studier //B definiera hur stora studier. Vi kanske annars kan ha som mål att det förhoppningsvis är till pedagogisk nytta.

Slutligen är det för oss inte uppenbart hur en kombination av fysik och domänspecifika språk kan se ut,
vilket är något vi nämner under projektets mål att vi vill undersöka.

\fi

% Vi kanske inte behöver skriva ut alla dessa, men de är ju däremot nyttiga för oss att internt fundera kring för att få stoff att besvara frågeställningen i rapporten.
%!! Frågor som vi ska både ställa och besvara.
%\begin{itemize}
%  \item Går det att implementera fysik m.h.a. DSL:er?
%  \item Kunde vi realisera våra initiala modeller (???)
%  \item Blir det ett bra läromaterial. (bra?)
%  \item Vad tyckte andra studenter      
%  \item Hur kan ett läromaterial för fysik med DSL se ut?
%\end{itemize}

\end{draft}

%Syfte/Mål

\section{Avgränsningar}

\begin{draft}

NYA

Läromaterialet ska begränsa sig till att enbart beskriva de fysikalsika områden som ingår i kursen Fysik för ingenjörer. Denna avgränsning valdes del för att det är den fysik gruppmedlemmarnas kunskapar begränsar sig till, dels för att det är för den kursen detta projekt kan bli mest relevant för, då kursen ingår i datastudenternas obligatoriska kursplan.

Vidare kommer en prioritering av innehållet i Fysik för ingenjörer att göras. Kursen behandlar grunderna inom mekanik (inklusive stelkroppsmekanik), termodynamik och vågrörelselära. Det ingår även en stor mängd tillämpad matematik, exemeplvis infinitesimalkalkyl. I första hand kommer mekaniken behandlas, för att sedan i mån av tid även behandla termodynamik och vågrörelselära. Fokuset kommer också främst att vara på de områden datastudenter haft svårt för.

För att utvärdera den pedagogiska nyttan kommer enbart en informell utvärdering att göras med en testgrupp. Detta då en rigorös undersökning hade krävt mycket tid för att välja lämpliga testgrupper, analysera återkopplingen samt dokumentera testningsförloppet.

GAMLA

Läromaterialet begränsar sig till att enbart beskriva en del områden inom fysik som ingår i kursen Fysik för ingenjöer. Denna avgränsningen valdes dells för att det är den fysik gruppmedlemmarnas kunskaper begränsar sig till, samt att det är den nämda kursen detta projekt kan bli mest relevant för, då kursen är en del av datastudenternas obligatoriska utbildning.

%TODO: PaJa: Jag tror att det i praktiken är en hel del tillämpad matematik inbäddat i kursen också, och det kan vara värt att nämna.
\textit{Fysik för ingenjörer} behandlar grunderna inom de tre områdena mekanik (inklusive stelkroppsmekanik), termodynamik och vågrörelselära. Vi har valt att i första hand prioritera mekanik, vilket även innefattar tillämpad matematik i form av Euklidisk geometri och infinitesimalkalkyl. Dessutom har de områden där datateknologer haft svårigheter prioriterats. I mån av tid har vi valt att behandla termodynamik och vågrörelselära (WE WILL SEE). 

Läromaterialet kommer testas inofficielt av en testgrupp. Detta ty en formell undersökning hade krävt omfattande tid för att välja ut en testgrupp på ett lämpligt sätt, analysera återkopplingen, samt dokumentering av testningsförloppet. Även om projektets direkta syfte inte är att förbättra tentastatistiken på \textit{Fysik för ingenjörer}, är vår förhoppning att förståelsen för de svåra delarna ska bli bättre.
\end{draft}

\begin{draft}
Vilka är projektet relevant för:
% Relevant för  ??? var bör relevant för vara? I syfte eller avgränsningar?
Projektet är relevant för datateknologer som läser en fysikkurs. Men
det kan också bli relevant för en fysikstudent som är ute efter en
inkörsport till funktionell programmering. Förhoppningsvis blir det
också relevant för de som är intresserade av domänspecifika språk i
stort, pedagoger och föreläsare inom de berörda områdena och kanske
till och med programledningen som ser vår rapport som ett skäl att
introducera innehåll av detta slag i till exempel fysikkursen.

\end{draft}

