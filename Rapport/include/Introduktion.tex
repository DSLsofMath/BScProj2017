
\chapter{Introduktion}

Detta kapitel beskriver projektets bakgrund, mål och avgränsningar.

\section{Bakgrund}

På civilingenjörsprogrammet Datateknik på Chalmers tekniska högskola ingår den
obligatoriska fysikkursen \textit{Fysik för ingenjörer}. Tentastatistiken för
denna kurs~\cite{tentastatistik} är ganska dålig\footnote{Andel underkända på ordinarie tentamen från läsår 2014 till
2017: 34\%, 76\%, 18\%, 57\%.}. Vi tror att många studenter på
Datateknik tycker att denna kurs är svår eller ointressant och att detta leder
till att en betydande andel får underkänt.

Detta tror vi kan lösas med avstamp från kursen \textit{Domain
Specific Languages of Mathematics} (``DSLsofMath''), med den svenska titeln
\textit{Matematikens domänspecifika språk}. Kursen är valbar på kandidatnivå för studenter på Chalmers och Göteborgs universitet. Konkret
presenterar DSLsofMath matematik som derivator, komplexa tal och
matriser ur ett programmeringsperspektiv i Haskell, vilket är ett programmeringsspråk datastudenterna redan är bekanta med.

DSLsofMath-kursens skapare, Cezar Ionescu och Patrik Jansson, har beskrivit avsikten med kursen i en artikel~\cite{tfpie2015}. Det direkta målet med kursen är
att förbättra den matematiska utbildningen för datavetare och den
datavetenskapliga utbildningen för matematiker, där den grundläggande idén
bakom kursen är:

\begin{center} ``[\dots] att uppmuntra studenterna att närma sig matematiska
  domäner från ett funktionellt programmeringsperspektiv: att ge beräkningsbevis
  (calculational proofs); att vara uppmärksamma på syntaxen för matematiska
  uttryck; och, slutligen, att organisera de resulterande funktionerna och
typerna i domänspecifika språk.''~\cite{tfpie2015}
\end{center}

Det programmeringsperspektiv som kursen använder sig av bottnar i
så kallade domänspecifika språk. Kortfattat kan ett domänspecifikt språk
beskrivas som ett programmeringsspråk som skapats för ett väl avgränsat
område. Detta område kan vara databashantering, algebraiska uttryck eller till
och med fysik. Språket kan antingen vara implementerat inuti ett annat
programmeringsspråk eller implementerat helt fristående. I kursen och projektet
är de implementerade i Haskell.

Idén bakom projektet är att använda domänspecifika språk för att ur ett alternativt perspektiv presentera fysik. Likt det sätt DSLsofMath
presenterar kopplingar mellan matematik och programmering ska projektet på motsvarande sätt visa på kopplingar mellan programmering och fysik och därmed
underlätta lärandet. För att förtydliga ges här en analogi:

%PaJa:Trevligt!

\begin{center}
Studenterna har svårt för matematik $\rightarrow $ DSLsofMath.\\
Studenterna har svårt för fysik $\rightarrow $ Detta projekt.
\end{center}

Detta projekt kan vara av intresse för studenter, pedagoger och
föreläsare inom de berörda områdena eftersom projektet ger ett nytt
perspektiv på fysik som inte bara är annorlunda utan också mer rigoröst.
Förhoppningsvis blir det även relevant för de som är intresserade av
domänspecifika språk i stort och kanske till och med för programledningen som
kan se denna rapport som ett skäl att introducera innehåll av detta slag i
fysikkurser.

Angående tidigare forskning och studier har en kurs på MIT, inte helt olik
DSLsofMath, tidigare givits som berör både fysik och
domänspecifika språk.
%funktionell programmering.
\textit{Classical Mechanics: A Computational Approach} gavs av Gerald Sussman
och Jack Wisdom senast år 2008~\cite{classical-mechanics-course-mit-2008}.
Denna kurs på avancerad nivå behandlar de fundamentala principerna för klassisk
mekanik med hjälp av beräkningsidéer för att precist formulera principerna av
mekanik, med början i Lagranges ekvationer och avslut i perturbationsteori
(teori för approximationer av matematiska lösningar). I kursens bok~\cite{SICM}
förklaras fysikaliska fenomen genom att visa datorprogram för att simulera dem,
skrivna i språket Scheme. Denna typ av kurs är ovanlig och är, till
projektgruppens kännedom, den enda kursen bortsett från DSLsofMath som knyter
samman fysik, programmering och matematik på en symbolisk nivå för att förklara
koncepten.

Även tidigare har det genomförts ett kandidatarbete på Chalmers med anknytning till DSLsofMath.
Vårterminen 2016 genomfördes kandidatarbetet \textit{Programmering som
undervisningsverktyg för Transformer, signaler och system. Utvecklingen av
läromaterialet TSS med DSL} av fem studenter från Datateknik och Teknisk
Matematik på Chalmers~\cite{kandidat2016}. Arbetet bestod av utveckling av läromaterial med
tillhörande programmeringskod, uppgifter och lösningar, som komplement till
existerande kurser i signallära.

Till sist finns det ett arbete som liknar detta arbete i både syfte och
programmeringsspråk, vilket utfördes av Scott N. Walck vid Lebanon Valley
College~\cite{lebanon-physics}. Syftet med det projektet var att fördjupa
studenters förståelse av fysik, med fokus på elektromagnetisk teori, genom att
uttrycka fysiken med hjälp av funktionell programmering.

\section{Projektets mål}

Målet med detta kandidatarbete är att angripa fysik från ett
programmeringsperspektiv, förhoppningen är då att fysik ska bli både
roligare och intressantare för datastudenter, och därmed också
enklare. Detta liknar premissen bakom kursen DSLsofMath och kandidatarbetet
från 2016, som istället för fysik behandlade matematik respektive signallära.

Mer konkret ska målet ovanstående uppnås genom att skapa ett läromaterial.
Läromaterialet ska bestå av
domänspecifika språk som modellerar fysik, skrivna
%programkod skriven
i Haskell,
sammanvävt med en förklarande lärotext. Läromaterialet ska vara
enkelt för läsaren att ta till sig, vilket ska åstadkommas genom ett lättsamt
språk, publicering på en hemsida samt fri tillgång till källkoden.

Ett parallellt mål är att, efter att ha tillägnat sig erfarenhet, diskutera
huruvida fysik och domänspecifika språk går att kombinera och om det finns en
pedagogisk nytta i att göra det.

\section{Avgränsningar}\label{sec:avgransningar}

Läromaterialet begränsas till att enbart hantera de fysikaliska områden
som ingår i kursen Fysik för ingenjörer. Denna avgränsning valdes dels eftersom
det är den fysik gruppmedlemmarnas kunskaper är begränsad till, och dels för att
det är denna kurs som projektet kan bli mest relevant för, eftersom
kursen ingår i datastudenternas obligatoriska kursplan.

Vidare ska en prioritering av innehållet i Fysik för ingenjörer göras.
Kursen behandlar grunderna inom klassisk mekanik, termodynamik och
vågrörelselära samt en stor mängd tillämpad matematik, exempelvis
differentialkalkyl. I första hand behandlas mekaniken, för att sedan i mån
av tid även behandla termodynamik och vågrörelselära. Fokuset läggs även
på de områden datastudenter haft svårt för.

För att pröva den pedagogiska nyttan kommer enbart en informell testning
att göras. Detta då en rigorös undersökning hade krävt mycket tid för att välja
lämpliga testgrupper, analysera återkopplingen samt dokumentera
testningsförloppet. Denna tid läggs istället på att skapa ett intressant innehåll.

Projektet fokuserar mer på att skapa innehåll än att göra
efterforskningar på, och tillämpa, pedagogiska teorier och riktlinjer. Denna
avgränsning valdes eftersom det är hur innehållet kan se ut som är
intressant och nytt, inte hur ett pedagogiskt läromaterial kan skrivas på bästa sätt. Den
pedagogiska aspekten kommer inte ignoreras helt, fokuset på den kommer bara att
vara mindre.
