% CREATED BY DAVID FRISK, 2016

\chapter{Introduktion}

\section{Bakgrund}

% Varför
På civilingenjörsprogrammet Datateknik på Chalmers ingår fysikkursen
\textit{Fysik för ingenjörer}. Tentastatistiken för denna kurs är
inte jättebra\cite{tentastatistik}. Vi
tror att många studenter på datateknik (``datateknologer'') finner denna kurs
svår eller ointressant, och att detta leder till att ungefär en
tredjedel av kursdeltagarna får underkänt på tentamen.

Examinatorn för kursen ``Fysik för ingenjörer, TIF085 (2016)'' Åke Fäldt, tycker
att studenter i allmänhet verkar ha svårt för att sätta upp egna
modeller. De baserar sina mentala modeller helt eller delvis på
intuition och felaktiga antaganden, istället för definitioner och
bevisade satser som man är säker på gäller. Detta leder till att de
tar genvägar som ofta är fel.

Sedan våren 2016 har kursen ``Domain Specific Languages of
Mathematics'' (``DSLsofMath'') eller ``Matematikens domänspecifika språk''
givits som en valbar kurs på kandidatnivå för studenter på Chalmers och
Göteborgs Universitet. År 2016 var Cezar Ionescu huvudföreläsare, och från 2017
är Patrik Jansson huvudföreläsare. Det direkta målet är att förbättra den
matematiska utbildningen för datavetare och den datavetenskapliga utbildningen
för matematiker, där den grundläggande idéen bakom kursen är: 

\begin{center}
  ``[\dots] att uppmuntra studenterna att närma sig matematiska domäner från ett
  funktionellt programmeringsperspektiv: att ge beräkningsbevis (calculational
  proofs); att vara uppmärksamma på syntaxen för matematiska uttryck; och,
  slutligen, att organisera de resulterande funktionerna och typerna i
  domänspecifika språk.''\cite{lecture-notes}\cite{tfpie2015} 
\end{center}

Konkret så presenterar kursen matematik så som derivator, komplexa
tal och matriser ur ett funktionellt programmeringsperspektiv i
Haskell. Dessa för studenterna bekanta verktyg används för att lösa
matematiska problem så som modellering av syntax, evaluering till
semantiska värden och datorassisterad bevisföring.

Även på MIT har en kurs inte helt olik DSLsofMath tidigare givits som berör både
fysik och domänspecifika språk (''DSL'').\ ``Classical Mechanics: A Computational Approach'' gavs av
Prof.\ Geral Sussman och Prof.\ Jack Wisdom bl.a. år
2008.\cite{classical-mechanics-course-mit-2008}
Denna kurs på avancerad nivå studerar de fundamentala principerna av klassisk
mekanik med hjälp av beräkningsidéer för att precist formulera principerna av
mekanik, med början i Lagranges ekvationer och avslut i
perturbationsteori. I kursen används boken ``Structure and
Interpretation of Classical Mechanics'' av Sussman, Wisdom och Mayer,
vilken förklarar fysikaliska fenomen genom att visa datorprogram för att
simulera dem, skrivna i språket Scheme.\cite{SICM}

Utöver DSLsofMath-kursen har det även tidigare gjorts ett kandidatarbete om DSL 
här på Chalmers. Vårterminen 2016 utfördes kandidatarbetet
``Programmering som undervisningsverktyg för Transformer, signaler och
system. Utvecklingen av läromaterialet TSS med DSL'' av fem studenter
från Datateknik och Teknisk Matematik på Chalmers. Arbetet bestod av
utveckling av läromaterial med tillhörande programmeringskod,
uppgifter och lösningar, som komplement till existerande kurser i
signallära.\cite{kandidat2016}

Vår tanke med detta kandidatarbete är att, likt premissen bakom DSLsofMath
och kandidatarbetet från 2016, med hjälp av ett bekant verktyg
som Haskell angripa fysik på ett sådant sätt att ämnet blir både
intressant och roligt för datateknologer, och därmed förhoppningvis
också enklare. Med hjälp av domänspecifika språk skrivna i Haskell för att
modellera fysik, i.e.\ samma pedagogiska verktyg som används inom
datakurser, tror vi att kopplingen mellan fysikkursen och
datateknikprogrammet kan göras tydligare och lärandet kan underlättas.
Förhoppningen är att bl.a.\ det kraftfulla typsystemet i Haskell ska
hjälpa studenter att bygga mentala modeller som är korrekta och inte
bygger på felaktiga intuitioner.

% Relevant för
Projektet är relevant för datateknologer som läser en fysikkurs. Men
det kan också bli relevant för en fysikstudent som är ute efter en
inkörsport till funktionell programmering. Förhoppningsvis blir det
också relevant för de som är intresserade av domänspecifika språk i
stort, pedagoger och föreläsare inom de berörda områdena och kanske
till och med programledningen som ser vår rapport som ett skäl att
introducera innehåll av detta slag i till exempel fysikkursen.

För läsaren som inte är insatt i domänspecifika språk, kan det
förklaras som ett språk konstruerat för ett specifikt område, en
domän. Språket kan användas för att enklare uttrycka saker inom
domänen, till exempel Newtons andra lag $F=m \cdot a$ än vad som är
möjligt inom generella (programmerings) språk. För vidare läsning
rekommenderas \textit{DSL for the Uninitiated}.\cite{DSLU}

\section{Rapportens syfte}

\begin{draft}
Rapportens syfte är att beskriva utvecklingen av läromaterialet, läromaterialet
i sig samt den tekniska bakgrund som krävs för att förstå läromaterialet. Vi går in på
svårigheter projektgruppen stött på under utvecklingen av läromaterialet, vilka områden
av fysik som lämpat sig väl samt lämpat sig mindre väl för implementering i ett 
domänspecifikt språk med syfte att tjäna som läromaterial.
%Ska vi berätta om:
%    att syftet är att rapporten utgör betygsgrundande material
%    att belysa hela projektets helhet (nu är det restriktivt till läromaterialet)
%    Ska vi beskriva vår ursprungsposition? fast det skulle kunna vara mer i bakgrund?  

%Rapportens syfte är att beskriva utförandet av projektarbetet samt utgöra betygsgrundande material för projektgruppen.
\end{draft}


\section{Projektets mål}

\begin{draft}

\iffalse
Projekets mål är att skapa ett roligt läromaterial som kombinerar fysik med
tillhörande domänspecika språk. Läromaterialets syfte är att väcka intresse hos läsaren
för fysikaliska samband samt inspirera till vidare studier. DSL:erna är menade att modellera specifikt utvalda
områden inom fysik. Den tillhörande brödtexten är menad att beskriva både DSL:erna, fysiken i sig,    %DSL:erna är bruten svenska?
samt kopplingen mellan dem.
\fi
% Annan formulering från nån annanstans
Målet med projektet är att skapa domänspecifika språk för fysik samt ett   %Målet med...
tillhörande läromaterial. Läromaterialets syfte är att beskriva fysiken och  %Läromaterialets syfte är ...
dess koppling till de domänspecifika språken projektgruppen utvecklat. Förhoppningen är att väcka intresse
för fysik hos datateknologer genom att presentera fysik från ett annat
perspektiv.

Läromaterialet är menat att i slutändan bestå av en hemsida. Källkoden för projektet
skall vara offentligt tillgänglig.

\end{draft}

\section{Avgränsningar}

\begin{binge}

Att börjar med mekanik o.s.v.

Att inte gör en ordentlig undersökning med testgrupp.

Läromaterialet begränsar sig till att enbart beskriva en del områden inom fysik som ingår i kursen Fysik för ingenjöer. Denna avgränsningen valdes dells för att det är den fysik gruppmedlemmarnas kunskaper begränsar sig till, samt att det är den nämda kursen detta projekt kan bli mest relevant för, då kursen är en del av datastudenternas obligatoriska utbildning.

%TODO: PaJa: Jag tror att det i praktiken är en hel del tillämpad matematik inbäddat i kursen också, och det kan vara värt att nämna.
\textit{Fysik för ingenjörer} behandlar grunderna inom de tre områdena mekanik (inklusive stelkroppsmekanik), termodynamik och vågrörelselära. Vi har valt att i första hand prioritera mekanik, vilket även innefattar tillämpad matematik i form av Euklidisk geometri och infinitesimalkalkyl. Senare, i mån av tid, har vi valt att även behandla termodynamik och vågrörelselära. Dessutom kommer de områden datateknologer haft svårt för prioriteras. Även om projekets direkta syfte inte är att förbättra tentastatistiken på \textit{Fysik för ingenjörer}, är vår förhoppning att förståelsen för de svåra delarna ska bli bättre.

\end{binge}

\section{Frågeställningar}

\begin{binge}

Kanske ska de vara här?

Ska de va punktform? Är detta faktiska frågeställningar eller höftade?

!! Frågor som vi ska både ställa och besvara.
\begin{itemize}
  \item Går det att implementera fysik m.h.a. DSL:er?
  \item Kunde vi realisera våra initiala modeller (???)
  \item Blir det ett bra läromaterial. (bra?)
  \item Vad tyckte andra studenter      
  \item Hur kan ett läromaterial för fysik med DSL se ut?
\end{itemize}

\end{binge}

