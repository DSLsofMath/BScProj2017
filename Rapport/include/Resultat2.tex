
\chapter{Resultat om domänspecifika språk och fysik}

\begin{draft}

I detta kapitel redovisas resultaten från möten med testgrupp och Åke Fäldt. De diskuteras och tolkas utförligare i nästa kapitel, kapitel \ref{cha:disk2}.

\section{Utvärderingen med testgruppen}

TODO: Anapassa till det nya syftet.

Utfallet från utvärderingen med testgruppen var till övervägande del positivt. Testgruppen tyckte läromaterialet var ett intressant och roligt sätt att presentera fysik på. De tyckte att bilderna tjänade sitt syfte, att muntra upp läsaren. 

En poäng som framfördes var att inte börja kapitlena för komplicerat. Istället tyckte de att det skulle vara bra att börja enkelt, för att kunna hänga med i både Haskell-koden och fysiken, för att därefter behandla ett område mer detaljerat. Att visa ett kort exempel i Haskell för att sedan låta läsaren själv göra något liknande var ett förslag på hur det kunde göras.

Utvärderingen var dock för kort för att det skulle framgå huruvida läsaren lärde sig mest fysik eller mest Haskell. Det framgick heller inte om läromatarialet uppmuntrade testgruppen att vilja lära sig mer fysik.

\section{Möten med Fäldt}

Åke Fäldt hade en överlag positiv syn på läromaterialet.\footnote{Det bör påpekas att det som är återgivit här självklart har tolkats, och kan ha missuppfattats, av projektgruppen. Fäldt ska med andra ord inte behöva stå till svars för vad som står här.} Fäldt tyckte att det fanns flera saker läromaterialet kunde bidra med. En bra sak var att läromaterialet ger att annat perspektiv på fysiken, ett annat sätt att förklara den genom att göra det med hjälp av domänspecifika språk.

En annan bra sak var den rigorösitet som domänspecifika språk leder till. Eftersom de domänspecifika språken måste vara väldefinerade betyder det att alla fysikalaiska koncept måste göras entydiga och även de blir väldefinerade. Operationerna på dem kan enbart göras på det definerade sättet. Följden blir att inget fusk kan göras i beräkningarna - alla steg måste vara fullständiga och följa de regler som finns. Fäldt menade att det var en bra egenskap hos läromaterialet, att detta rigorösa tankesätt och metodik som förmedlas hade varit till nytta i problemlösning i fysikkursen.

Förutom ovanstående framgick även vilka områden i Fysik för ingenjörer som var svåra för studenter. Detta finns redovisat i avsitt \ref{sec:kontakt_faldt}

\end{draft}






























