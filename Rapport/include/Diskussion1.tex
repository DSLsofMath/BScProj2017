
\chapter{''Övrig'' Diskussion}

\begin{binge}
De tre nedanstående ska vara nånstans i diskussion.

De domänspecifika språken modellerar områden snarare än att vara problemlösare.
\textit{Analys} exemplifierar detta väl. Det språket består av ett syntaxträd
över algebraiska uttryck samt operationer som derivering och integration. Med
hjälp av det kan man modellera uttryck och analytiska operationer på dem.
Däremot löser det inte problem åt en. Man kan med andra ord inte mata in en
differentialekvation och automatiskt få en lösning.

Hur bra våra moduler blev samt varför de inte är problemlösare utan modellerare istället.

När områden kodades upp till domänspecifika språk var det långt ifrån alla gånger
det kunde göras på en meningsfullt sätt \todo{Expandera, Vad betyder meningsfullt?}.

\section{Teori- och metodval}

Här ska vi vara kritiska till vårt teori- och metodval.

Sökande: När vi valde områden och gjorde läromaterial borde vi ha tydligt listat de olika ''avsnitten'' i fysikkursen, och gjort ett DSL efter dom. Istället har vi valt områden lite på måfå efter vad som verkat enklast att göra. Borde planerat bättre för att anpassa de grundläggade DSLerna till vad som ingår i fysikkursen. Så att mer relevant material och mer jämnt täckande av hela kursen. Har lite att göra med avsnitt 1 i diskussion2.

Fast allt det ovanstående kanske är inaktuellt, för Åke sa ju att det säkert hade räckt med ett par ordentliga områden, så att tankesättet förmedlas.

Skrivande: Pedagogsik aspekt i teori och metod bristande. Borde i teori gjort mer allmän inläsning på pedagogik i allmänhet för vi kunde inte så mycket i början. Sedan borde ha lagt större vikt på att pedagogiskt riktigt i metod när vi skapade materialet. Svårt att veta om DSL+fysik pedagogiskt om materialet i sig är dåligt pedagogiskt. DSL2016 skrev om interaktiv pedagogik. Hade vi kunnat ha med eller liknande.

Utvärderingar: I samband med ovanstående, både om läromaterialet roligt/lärorikt och om DSL+fysik pedagogiskt, borde haft större och längre utvärdering. I allmänhet få möten och utvärderingar. Se vidareutveckling, om att kunde ha större undersökningar...

--------

Detta ska anpassas in nånstans i det ovanstående

Fanns mycket i ARCS vi inte använder. Kanske därför inte kan förvänta oss att så bra pedagogiskt.

Men projekets fokus var snarare av teknisk karaktär.

Egentligen har det väl varit skapandet som varit det lärorika och inte användingen av det vi gjort? DSL2016 tyckte liknande här, så kan hänvisa till dom.

Kan även hänvisa till dom att de la mycket tid på didaktik så mindre tid på själva läromaterialet.

\section{Läromaterialet}

En stor del av projektets mål var att skapa ett läromaterialet som både skulle
vara roligt och intressant att läsa. Vi skulle åstakomma detta genom att använda
oss av ARCS-metoden (?), användandet av lättsam text och roliga bilder för att
lätta upp materialet. 

\section{Vidareutvecklingsmöjligheter}

Läromaterialet innehåller domänspecifika språk för de \textit{matematiska} områdena analys och vektorer. Dessa områden används sedan för att koda upp och lösa uppgifter av mer \textit{fysikaliska} slag, till exempel lutande plan. Med andra ord görs inga domänspecifika språk för fysik i sig. En vidareutveckling hade därmed varit att göra precis det, att inte bara tillämpa matematiska domänspecifika språk utan göra fysikalsika domänspecifika språk. Det kan vara saker som ett syntaxträd för ett lutande plans komponenter. Det kan vara ett syntaxträd för vilka krafter som verkar på fysikalsika kroppar i mekanikproblem. Det kan till och med vara ett domänspecifikt språk för något så abstrakt som fysikalisk problemlösning i allmänhet. Vi har dessvärre ingen aning hur det skulle kunna se ut. Men av just detta skäl tror vi det hade varit väldigt intressant att se hur ett mer fysik-orienterat domänspecifikt språk kan se ut.

En annan möjlig vidareutveckling är att göra en rigorös studie kring de pedagogiska aspekterna kring kombinationen av fysik och domänspecifika språk. Detta projekt innehöll enbart en mindre sådan studie. Det som kan vara intressant att undersöka är om studenter tycker att fysik blir intressantare genom en kombination av detta slag och kanske därför studerar mer i fysikkursen. Eller kanske om de rent av blir bättre på fysik i sig genom att fysik presenterats på detta sätt, det vill säga att ett läromaterial av detta slag skulle kunna fungera istället för traditionell undervisning inom fysik. Givetvis skulle kompletering av läromaterialet behöva göras så att det är heltäckande.

Slutligen kan det befintliga läromaterialet byggas vidare på. I sin nuvarande form behandlas varken termodynamik eller vågrörelselära något alls. Dessutom lär det finnas aspekter inom den klassiska mekaniken som fattas.

\section{Etiska aspekter}

En bakomliggande tanke vi haft genom hela projektet är att läromaterialet ska vara tillgängligt för alla. Därav har vi valt att publicera det på en hemsida. Denna hemsida använder grundläggande HTML och CSS samt javascript. Javasript är dock inget krav för funktionalitet. Eftersom hemsidan är lättviktig bör den fungera väl även på gamla datorer och telefoner, till skillnad från tunga PDF-filer och många moderna hemsidor.

Tanken om tillgänglighet ligger även bakom valet att låta källkoden var fritt tillgänglig. Visserligen \textit{är} läromaterialet i princip hela källkoden, så har man läromaterialet har man källkoden. Att ha källkoden dirket har dock fördelar som att man kan följa med i versionshistoriken, man kan se kommentarer och alternativa implementationer som inte syns i den slutgiltiga produkten samt att det blir enklare att modifera källkoden och man inte behöver reverse-enginnera hemsidan. Det handlar om att visa att man är positiv till att andra tittar hur man gjort och låta andra bygga vidare på ens skapelser. Genom att sluta oss till skaran som tillämpar öppen källkod hoppas vi att fler inom samhället i stort ska gå över till denna modell.

Valet att skriva på engelska har också att göra med tillgängligheten. Fler kan engelska än svenska. Så på detta sätt kan läromaterialet komma fler till gagn.

Roligt <-- vet ej hur ska få in bra.



Om Hemsidan:
  Det är lite intressant ur pedagogik-aspekten. Kan det
  kanske vara lättare/roligare att lära sig om sidan är fin och
  lättläst? Att javascript inte krävs gör att sidan kan visas ordentligt
  även om man sitter i U-land med dålig/gammal/billig telefon.


\end{binge}
