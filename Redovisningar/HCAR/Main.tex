\documentclass[DIV16,twocolumn,10pt]{scrreprt}
\usepackage{paralist}
\usepackage{graphicx}
\usepackage[utf8]{inputenc}
\usepackage[final]{hcar}

%include polycode.fmt

\begin{document}

\begin{hcarentry}{Learn You A Physics}
\report{Erik Sjöström}
\status{active development}
\participants{Oskar Lundström, Johan Johansson, Björn Werner}% optional
\makeheader

Put the text here. 
If you want to include Haskell code, consider using lhs2tex syntax (\url{http://people.cs.uu.nl/andres/lhs2tex/}).

(WHAT IS IT?)

Learn You A Physics is the result of a bachelor's project (at) where the
goal is to create a learning material for physics aimed at programmers
with a basic understanding of Haskell. 

It does this by identifying key areas in physics with a well defined scope,
for example dimensional analysis or single particle mechanics, and develops
a domain specific language around this area.

The implementation of these DSL's are the meat of the learning material with
accompanying text to explain every step and how it relates to the physics of
that specific area. 

The text is written in such a way as to be as non-frightening as possible, 
and to only require a beginner knowledge in Haskell.

Inspiration is taken from länk:LYAH and the project DSLsofMath at ch gu

What's following are suggestions for the content of an entry.

(EXEMPELBILD PÅ MATERIALET?)

(WHAT IS ITS STATUS? / WHAT HAS HAPPENED SINCE LAST TIME?)

(CAN OTHERS GET IT?)

The \href{https://github.com/DSLsofMath/BScProj2018/tree/master/Physics}{source
code} and \href{https://dslsofmath.github.io/BScProj2018/}{learning material}
is freely available online.

(WHAT ARE THE IMMEDIATE PLANS?)

\FurtherReading
  \href{https://dslsofmath.github.io/BScProj2018/} {Learn You A Physics}
\end{hcarentry}

\end{document}
